\chapter{结论}
本研究围绕 AI 生成的个性化广告展开,系统性探讨了 AI 在不同人格特质、匹配方式和广告创作模式下的有效性,以及 AI 作为信息源对个性化广告接受度的影响。在研究一和研究二中,我们首先验证了 AI 生成的个性化广告在不同人格特质群体中的适用性,并比较了 AI 与人类专家在个性化广告创作中的表现。研究结果表明,AI 在开放性和外倾性广告的个性化匹配上较为稳定,但在尽责性和宜人性维度的匹配效果存在不稳定性,神经质维度的匹配效应整体不显著。此外,AI 在逻辑性和结构化表达上较人类专家更具优势,但在人类专家更擅长的情感共鸣型广告(如宜人性广告)中,AI 仍然存在一定局限。研究三进一步分析了 AI 生成广告的语言特征,发现 AI 生成的个性化广告主要基于表层语言风格的匹配,而未能充分捕捉受众在广告语境中的深层需求偏好。研究四探讨了 AI 作为信息源对个性化广告接受度的影响,发现 AI 作为广告创作者的身份披露会降低广告的说服力,这种影响与受众的感知相似性有关。本研究的发现不仅揭示了 AI 在个性化广告创作中的优势与局限,也为未来优化 AI 生成广告的策略提供了理论与实践指导。

当前研究得到以下主要结论:

1. AI 生成的个性化广告在不同人格特质上的匹配效果存在差异。开放性和外倾性广告的个性化匹配效果较为稳定,而尽责性和宜人性广告的匹配效果受特质水平和广告创作方式的影响。神经质个性化广告的匹配效应整体不显著。

2. AI 与人类专家在个性化广告创作中的表现各有所长。AI 在逻辑性强、结构化的信息传递上更具优势,尤其在尽责性广告的匹配效果上优于人类专家。然而,在强调情感共鸣的广告(如宜人性广告)中,人类专家的表达能力仍然占优。

3. AI 生成的个性化广告在语言风格上能够与目标受众保持一致,但其个性化匹配策略较为表层,未能充分挖掘广告语境下受众的深层次需求。个性化广告的优化需要结合受众的认知偏好,而非仅依赖语言风格的匹配。

4. AI 作为信息源对个性化广告的接受度具有显著影响。实验表明,受众在未知广告来源的情况下,对 AI 生成广告的接受度较高,但当 AI 身份被明确披露后,广告的说服力下降。这一影响与受众的感知相似性有关,即 AI 作为广告创作者可能增加受众与广告之间的心理距离,降低个性化广告的信任感。

本研究拓展了 AI 在个性化广告中的应用研究,并提供了优化 AI 生成广告的策略,为个性化广告的理论研究和实际应用提供了新的视角。未来 AI 生成个性化广告的优化路径应包括三方面:一是通过优化提示词,使 AI 生成的广告不仅关注语言风格匹配,还能结合目标群体的认知加工偏好;二是结合人机协作的创作模式,让 AI 在优化人类专家创作的广告文本方面发挥作用;三是调整 AI 广告的信息披露方式,以降低 AI 作为广告创作者的负面认知影响,提高个性化广告的整体接受度。