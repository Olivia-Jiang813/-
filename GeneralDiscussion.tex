\chapter{总讨论}
近年来,生成式人工智能(Generative AI),尤其是 大语言模型(Large Language Models, LLMs),在文本生成领域展现出巨大的潜力。AI 不仅具备自主生成信息(如广告文本)的能力 \citep[例如][]{karinshak2023working, bai2023artificial},还能够根据目标受众的特征快速调整广告风格,从而增强个性化匹配度 \citep[例如][]{matz2024potential, simchon2024persuasive}。这一特性使得 AI 在个性化广告创作中展现出广阔的应用前景,然而,其生成的个性化广告是否真正有效仍需进一步验证。一方面目前尚不清楚 AI 是否能针对不同人格特质、不同匹配水平、不同生成方式,稳定地生成有效的个性化广告。另一方面,人们是否能区分 AI 与人类专家撰写的广告?如果 AI 的身份被明确告知,这种信息来源的认知是否会影响个性化广告的接受度?这些问题在现有研究中尚缺乏系统性的探讨。基于此,本研究围绕这些核心问题展开,采用四个子研究,系统性地探讨 AI 生成的个性化广告的有效性、适用性以及优化路径。

子研究 1 旨在验证 AI 生成个性化广告的有效性及其在人格匹配上的适应性。通过多个实验,本研究首先采用 GPT 生成针对大五人格不同特质的个性化广告,并评估其在不同人格特质群体中的说服效果。结果表明,AI 生成的个性化广告在 外倾性和宜人性群体 中表现出较强的匹配效应,而在 尽责性和神经质群体 中,匹配效应较为不稳定。此外,GPT-4 在个性化广告生成上的表现优于 GPT-3.5,尤其在 开放性和尽责性群体 中,其广告匹配度显著提高。进一步的实验探讨了 AI 在不同广告创作情境下的表现,发现 AI 不仅能够从零生成个性化广告,还能够基于现有广告文本进行个性化改写,表明其在个性化广告定制中的灵活性。

子研究 2 进一步比较了 AI 与人类专家在个性化广告创作中的差异,并探讨 AI 优化人类专家广告文本 的潜力。实验结果表明,AI 在多个特质(如尽责性)上的个性化效果甚至优于人类专家,但在人类创作较具优势的 宜人性广告 维度上,AI 仍存在一定局限。然而,当 AI 用于优化人类专家撰写的广告 时,其个性化广告的效果在宜人性维度上显著提升,这表明 AI 不仅能够独立生成个性化广告,还能够增强人类专家创作的广告的个性化匹配度。

子研究 3 采用文本分析与预测建模的方法,深入探讨 AI 生成的个性化广告文本与实际受众广告偏好之间的匹配性。通过对不同人格特质个性化广告的语言特征进行分析,研究发现,AI 生成的广告语言风格在大多数情况下能够与目标人格特质相匹配,但在特定特质(如神经质和低开放性个体)上,其语言风格匹配度较低,且在尽责性维度上出现 负向匹配效应,即高尽责性个体更偏好针对低尽责性设计的广告,低尽责性个体则更倾向于高尽责性广告。此外,研究发现 个性化广告的语言特征匹配与个体实际广告偏好之间并非完全一致,AI 生成的个性化广告在某些情境下可能过于依赖人格特质的表层语言风格,而忽略了目标受众在广告语境下的更深层次的需求偏好。

子研究 4 进一步探讨了 AI 作为信息源对个性化广告接受度的影响,重点考察 当受众知道广告由 AI 生成时,其对广告的态度是否发生变化。研究结果表明,受众能够基于语言风格判断广告创作者,但当 AI 作为广告创作者的身份被明确披露 时,个性化广告的说服力下降。这一现象表明,个性化广告的接受度不仅取决于广告内容本身,还受到受众对信息来源的认知影响。AI 作为信息源可能 增加受众与广告之间的心理距离,从而降低个性化广告的契合度,特别是在需要情感共鸣的广告类型中,AI 生成的广告可能因缺乏人类创作者的情感深度而影响受众的信任度。这一发现提示,在实际应用中,AI 生成的个性化广告需要在信息来源披露方式上进行优化,以减少 AI 作为广告创作者的潜在负面影响。

综合来看,本研究通过 四个子研究,系统性地探讨了 AI 生成的个性化广告的有效性、AI 在个性化匹配上的优势与局限、AI 生成广告文本的语言风格与实际受众偏好的匹配度,以及 AI 作为信息源对个性化广告说服力的影响。研究结果不仅验证了 AI 在个性化广告创作中的应用潜力,也揭示了 AI 在不同人格特质群体中的适应性差异,并提供了 优化 AI 生成广告内容及披露策略 的方向。

\section{AI生成的个性化广告的有效性及其局限}

本研究的结果表明,AI 生成的个性化广告在一定程度上展现出较强的说服力,但其有效性并不均衡,受不同人格特质、匹配水平和广告创作方式的影响。研究发现,AI 在外倾性和开放性个体中能够稳定生成符合受众偏好的广告,而在宜人性、尽责性和神经质个体中,其匹配效果较为不稳定,甚至在某些情况下完全失效。此外,AI 生成的广告在语言风格匹配上虽然能够与目标受众保持一致,但在实际广告偏好和说服力上,仍然存在一定的局限性。

首先,\textbf{AI 在不同人格特质群体中的有效性存在差异}。研究一的结果表明,开放性和外倾性是两个较为稳定的人格维度,无论是高水平还是低水平个体,AI 都能够生成符合其偏好的个性化广告文本,并在匹配度和说服效果上表现出显著优势。相比之下,宜人性和尽责性的个性化广告效果则较为不稳定。宜人性的高水平个体始终能够被有效匹配,而低水平个体的匹配效果取决于广告生成方式:在基于中性广告改编的条件下,AI 仍能生成有效的个性化广告,但在基于产品描述直接生成的条件下,个性化效果未能显现。尽责性个体则表现出相反的趋势,高尽责性个体能够被 AI 生成的广告有效匹配,而低尽责性个体始终未能在任何条件下展现出显著的匹配效应。此外,神经质是唯一一个在所有条件下均未能展现个性化匹配效应的人格维度,无论是高水平还是低水平个体,AI 生成的个性化广告在该维度上的效果均不显著。这些发现表明,AI 在生成个性化广告时,并非对所有人格特质均能实现稳定的匹配效果,不同人格特质的个性化适配性可能受到目标受众的特质特征、广告生成方式以及个性化策略本身的影响 \citep{winter2021effects}。进一步地,研究二的结果表明,AI 与人类专家在个性化广告创作中的表现亦存在显著差异。在尽责性广告的个性化匹配度上,AI 的表现甚至优于人类专家,而在人类更具优势的宜人性广告维度上,AI 仍然存在一定局限。这一发现表明,AI 在处理结构化、目标导向型的信息(如尽责性广告)时可能更具一致性,而在人类专家擅长的情感共鸣和社交互动类广告(如宜人性广告)中,AI 的表达方式相对较为刻板,缺乏人类创作者所具备的情感深度。此外,研究进一步发现,当 AI 用于优化人类专家创作的广告文本时,其个性化广告的整体说服效果在宜人性维度上得到了显著提升。这表明 AI 并非单纯地替代人类创作者,而是具备增强人类专家广告个性化表达的潜力。在实际应用中,这一结果提示,AI 生成的个性化广告可以作为辅助工具,与人类创作者协同工作 \citep{yoon2024designing},以结合 AI 的高效性和人类创作者的情感共鸣能力,从而优化个性化广告的整体表现。

其次,\textbf{AI 生成的个性化广告在不同人格特质上的匹配效果存在局限性,主要体现在其对受众需求的理解较为表层化,未能充分结合广告语境中的深层次偏好。}研究三通过文本分析和预测建模,系统性比较了 AI 生成的个性化广告与用户的实际广告偏好,发现尽管 AI 能够在语言风格上模仿不同人格特质的表达方式,但其个性化策略往往停留在表层特征,而未能深入把握广告情境下受众的信息加工方式。例如,高开放性个体在日常交流中更倾向于使用抽象、探索性的语言风格,因此 AI 在生成个性化广告时,往往更关注这类语言特征的匹配。然而,在广告语境中,由于广告本质上是一种说服性传播,高开放性个体在深层加工过程中可能更关注因果逻辑与创新创意的结合,而非单纯的发散思维和抽象表达。换言之,AI 可能能够识别并复制开放性个体在一般语境中的语言风格,但在个性化广告场景下,其生成的内容未能充分结合广告的传播目标,导致个性化匹配的有效性受限。同样,宜人性个体的广告偏好在过往研究中未得到充分探讨,既有文献通常强调宜人性个体偏好和谐、温暖的沟通方式。然而,研究三的结果表明,高宜人性个体不仅关注广告的社交取向,还关注广告如何凸显产品在人际关系或个人体验中的独特性。例如,他们可能更倾向于看到产品如何促进人与人之间的联系,或如何体现个体在社群中的积极形象,而 AI 生成的个性化广告往往只是简单地强调情感共鸣,忽略了这些更深层次的社交价值表达。低宜人性个体则倾向于直接、批判性更强的广告风格,而 AI 生成的广告通常采用更温和、含蓄的表达方式,导致个性化匹配的有效性下降。此外,在神经质维度上,研究三的结果揭示了高神经质与低神经质个体在广告偏好上的表现形式相似但驱动因素不同。高神经质个体关注情绪共鸣,但更需要即时的情绪安抚,如非正式表达和即时满足;相比之下,低神经质个体对情绪化广告内容的接受度较高,但更偏好结构清晰、强调事实与稳定性的广告信息。这一发现表明,神经质个体的个性化广告设计不能仅依据其在日常语言中的情绪表达特征,而应结合个体在广告语境下的实际心理需求进行优化。

综上所述,AI 生成的个性化广告在不同人格特质群体、不同广告创作方式和不同语言风格适配度上,均展现出一定的有效性,但也存在局限性。研究一和研究二的结果表明,AI 在某些人格特质(如外倾性和宜人性)群体中能够有效地提升广告说服力,但在尽责性和神经质群体中,其匹配效应存在不稳定性。此外,研究三的文本分析进一步揭示了 AI 生成广告的语言风格与实际广告偏好的不完全一致性,说明个性化广告的优化需要超越语言特征匹配,进一步结合目标受众在广告语境下的认知方式、信息加工需求和心理调节机制。研究二的实验也表明,AI 生成广告在某些情况下可以有效地优化人类专家创作的广告文本,特别是在需要结构化、逻辑清晰的信息表达时,AI 可能更具优势,而在人类创作者更具情感表达能力的广告(如宜人性广告)中,AI 仍然面临一定的挑战。这些发现共同表明,AI 生成的个性化广告的有效性虽然得到了一定验证,但其适用性并非普适,未来的优化方向应结合认知科学与广告传播的多维视角,使个性化广告不仅能够匹配目标人格特质的语言风格,更能精准满足不同群体的实际广告接受模式。

\section{AI作为信息源对个性化广告的影响}
本研究进一步探讨了 AI 作为信息源对个性化广告接受度的影响。尽管 AI 在个性化广告创作中展现出一定的有效性,但广告的说服力不仅依赖于内容本身,也受到信息来源的影响。研究四的实验结果表明,当受众未知广告创作者身份时,他们能够基于语言风格对广告的来源做出一定推测,但这种判断并不总是准确,因此不会直接影响广告的说服力(研究二)。然而,当 AI 作为广告创作者的身份被明确披露后,受众对个性化广告的接受度整体下降,广告的说服力随之削弱,受众的信任度也降低。这一现象表明,个性化广告的有效性不仅取决于广告内容本身,还受到受众对 AI 作为信息源的认知影响。


\textbf{当广告来源未知时,受众可以在一定程度上区分 AI 生成的广告与人类专家撰写的广告,但这种区分能力并不稳定。}具体而言,GPT 生成的广告更倾向于被识别为 AI 生成的,而人类创作的广告更倾向于被认为是由人类撰写的。然而,实验结果也显示,部分 AI 生成的广告会被误认为是人类创作的,而部分人类创作的广告可能被误认为是 AI 生成的 \citep{chaka2024reviewing}。这一现象表明,受众对 AI 生成广告的辨别能力并不完全可靠,同时也反映了 AI 在某些情境下能够较好地模拟人类表达,使广告信息的来源变得模糊。这一发现强调了信息来源披露可能带来的影响,并提示未来在 AI 生成广告的实际应用中,需要更加谨慎地制定信息披露策略。此外,本研究的实验验证了感知相似性在个性化广告接受度中的作用。实验结果表明,当广告内容与受众的人格特质匹配时,受众对广告的感知相似性更高,进而增强广告的接受度。结合这一发现,本研究进一步探讨了 AI 作为信息来源的披露如何影响广告接受度,\textbf{并假设当 AI 作为广告创作者的身份被明确披露后,受众可能会认为 AI 并不能真正理解个体需求,从而增加受众与广告之间的心理距离\citep{kim2020artificial, ahn2021ai},进而削弱广告的整体说服力。}这一发现表明,在个性化广告的应用中,AI 生成的广告不仅需要语言风格的匹配,还需要综合考虑信息来源的影响,以降低 AI 作为广告创作者可能带来的负面效应。


这一发现对 AI 生成个性化广告的优化策略具有重要启示。一方面,优化 AI 生成广告的表达方式 可以减少受众对 AI 作为信息源的心理距离,从而提高广告的接受度。研究表明,使用第一人称和直接面向受众的语言能够增强个性化体验,使广告更加具有互动性,并提升其说服力 \citep{markowitz2020communicating}。此外,在强调情感共鸣和社交互动的广告类别中,AI 生成的文本可以融入更多自然的人类表达方式,例如加强幽默、情感化描述和叙述性元素,以降低 AI 生成文本的刻板印象,使其更符合人类创作的风格。研究二的对比实验表明,AI 在尽责性广告的个性化匹配上展现出较强的逻辑性和信息密度,比人类专家创作的广告更具说服力;而在宜人性广告维度上,GPT-4 通过优化人类专家撰写的广告文本,使个性化效果显著提升。这一结果表明,AI 生成的个性化广告不仅可以通过调整语言策略增强自身的接受度,同时也可以通过与人类专家的结合,进一步提升广告的个性化适配性。另一方面,调整 AI 生成广告的信息披露方式 也可以有效降低 AI 作为信息来源可能带来的负面影响。研究发现,当 AI 生成的内容被描述为“结合专家意见”或“基于大数据精准分析”时,其可信度和接受度有所提升 \citep{puerta2022human}。这表明,即使政策要求披露 AI 生成广告,企业仍可以通过优化披露方式来减少 AI 身份对广告接受度的负面影响。例如,可以在广告中强调 AI 在创作过程中的辅助角色,而非主要创作者,或采用人机共创(human-in-the-loop)模式,让人类专家优化 AI 生成的广告内容。这种方式不仅能够结合 AI 在数据分析和高效文本生成方面的优势,同时也能保留人类专家的创造力和情感共鸣能力,使广告既具备规模化生成的优势,又能保持个性化和真实性。

综上所述,AI 作为信息源对个性化广告的影响具有双重特性。一方面,AI 生成的广告在逻辑性和信息传递方面展现出一定优势,特别是在尽责性和开放性广告类别中,AI 可能比人类专家更具说服力;但另一方面,在强调情感共鸣和社交互动的广告类别中,AI 生成的广告在披露后可能会削弱受众的信任感,导致广告说服力下降。因此,在 AI 生成个性化广告的实际应用中,应综合考虑受众对信息来源的认知影响,优化 AI 生成广告的语言策略,并探索更灵活的信息披露方式,以最大程度地发挥 AI 在个性化广告创作中的优势,同时减少 AI 作为信息源可能带来的消极效应。

\section{个性化广告的优化路径}
本研究的结果表明,AI 生成的个性化广告在一定程度上展现出较强的说服力,但其有效性受多种因素影响,包括目标受众的人格特质、个性化匹配程度、广告创作方式以及信息来源的认知。在此基础上,优化 AI 生成的个性化广告,需要在理论层面进一步明确个性化广告的关键作用机制,并在实践层面结合 AI 提示词优化、信息来源管理及人机共创模式,以提升广告的个性化匹配效果和受众接受度。

在理论层面,本研究的核心贡献体现在个性化广告的匹配机制、AI 与人类专家的互补性以及信息来源认知对广告接受度的影响。首先,\textbf{个性化广告的匹配机制不仅依赖于表层语言风格,而是深层次的信息加工方式决定了个体对广告内容的偏好}。过往研究主要基于大五人格特质设计广告内容,但匹配效应的稳定性存在一定争议 \citep{matz2017psychological,winter2021effects}。本研究通过文本分析和实验发现,个性化广告效果的不稳定性可能源于不同人格特质个体在广告情境下的信息处理方式更具独特性。例如,高开放性个体在日常交流中偏好抽象、探索性的语言风格,AI 生成的广告也能模仿这一特点。然而,在广告语境中,高开放性个体更关注因果逻辑与创新创意的结合,而非单纯的抽象表达。类似地,高神经质个体在一般对话中展现出高情绪性的语言特点,但在广告中,他们更倾向于非正式表达和即时满足,而非强化负面情绪。因此,AI 生成的个性化广告不能仅依赖于语言风格的匹配,而需要进一步结合目标受众在广告情境下的实际信息加工方式。其次,\textbf{信息来源的认知影响也是个性化广告接受度的重要因素。}本研究的实验验证了感知相似性在个性化广告接受度中的中介作用,发现受众更倾向于接受与自身特质相似的信息来源。当广告创作者身份未知时,受众可以在一定程度上推测广告来源,但这种判断并不总是准确,因此对广告接受度的影响有限。然而,当 AI 作为广告创作者的身份被明确披露后,个性化广告的说服力下降。这一现象可能源于 AI 作为信息源所带来的心理距离,使得受众难以产生认同感,从而降低广告的可信度和影响力。特别是在强调情感共鸣的广告(如宜人性广告)中,受众更倾向于相信由人类专家创作的广告,而 AI 生成的广告则可能因缺乏情感真实性而削弱其说服力。此外,本研究还发现\textbf{ AI 与人类专家在个性化广告创作中各有所长,二者在不同广告类别中的优势互补}。在尽责性广告中,AI 生成的广告由于逻辑性强、信息密度高,能够精准匹配目标群体的需求,甚至优于人类专家的创作。然而,在宜人性广告中,人类专家的表达能力更具优势,能够更自然地融入情感共鸣和社交互动的元素,而 AI 生成的广告则可能显得较为刻板。因此,AI 在个性化广告中的最佳角色可能并非完全替代人类创作者,而是作为辅助工具,与人类专家协同工作,以优化广告创作效果。

基于上述理论机制,在实践层面,本研究提出了针对 AI 个性化广告的优化策略,包括优化 AI 语言风格的操控方式、采用人机结合的广告创作模式以及调整 AI 生成广告的披露方式,以提升个性化广告的整体效果。首先,AI 生成的个性化广告在现有实践中主要依赖于提示词优化,但仅关注表层语言风格可能无法达到最佳的说服效果。本研究的发现表明,优化 AI 生成广告的一个关键策略是改进提示词设计,使 AI 在创作过程中不仅关注受众的语言风格,还能结合目标群体的认知偏好。\textbf{进一步的优化手段包括基于少样本学习和微调技术,使 AI 在训练过程中融入更符合个性化广告需求的文本示例。}此外,利用研究三的预测建模方法,可以快速评估 AI 生成广告的有效性,帮助进一步优化提示词或生成模型,以提升广告的匹配度 \citep{raut2023reinforcing}。在人机结合的广告创作模式方面,本研究发现 AI 在情感共鸣较强的广告类别(如宜人性广告)中表现不及人类专家,但当 AI 用于优化人类专家撰写的广告文本时,个性化广告的整体说服效果得到了提升。\textbf{这表明,与其让 AI 独立创作广告,不如让 AI 在已有广告文本的基础上进行优化,以更符合受众的需求}。人机结合的广告创作模式不仅能够提升个性化广告的效果,还能够保留 AI 在高效生成和优化广告方面的优势,同时结合人类创作者的情感共鸣能力,使广告更具吸引力。在 AI 生成广告的披露方式方面,本研究发现,受众在未知广告来源的情况下,对 AI 生成的广告接受度较高,但当 AI 身份被明确披露后,广告的说服力下降。这一结果表明,优化 AI 生成广告的披露方式是降低 AI 负面认知的重要策略。例如,研究发现,当 AI 生成的广告被描述为“结合专家意见”或“基于大数据精准分析”时,其可信度和接受度显著提升。因此,即使政策要求披露 AI 生成广告,企业仍可以通过优化披露方式减少 AI 身份对广告接受度的负面影响。另一种可能的策略是采用人机共创的模式进行披露,即让 AI 生成的广告由人类专家审核和优化,并在披露信息中强调人类专家的参与,从而降低 AI 生成广告带来的心理距离。

总体而言,本研究的发现不仅拓展了 AI 生成个性化广告的理论基础,还为实践中的广告优化提供了可行的策略。从理论层面来看,本研究揭示了个性化广告优化不仅需要语言风格匹配,更需结合受众的深层需求、感知相似性以及 AI 作为信息源的影响。从实践层面来看,本研究提出了基于优化 AI 语言风格、人机结合创作以及调整披露方式的策略,以提升 AI 生成个性化广告的整体效果。未来,随着 AI 在广告行业中的应用不断发展,如何进一步优化 AI 生成的广告内容,使其既能保持高效性,又能增强受众的认同感和信任感,将成为个性化广告研究的重要方向。

\section{研究局限及未来研究方向}

本研究尽管对 AI 生成个性化广告的有效性和适用性进行了系统性探讨,但仍然存在一些值得进一步优化和扩展的局限性。可以从以下几个方面进行梳理:个性化匹配机制的复杂性、数据特征的局限性、广告形式的单一性、因变量的现实贴合度、以及理论基础的进一步拓展。每一方面的局限性不仅影响了当前研究结论的适用性,也为未来研究提供了新的探索方向。

首先,关于大五人格特征的个性化设计,本研究主要基于单一维度进行设计和实验检验。但实际上,个性化广告的有效性可能涉及多个特质的交互作用。在实验过程中,本研究发现,尽管低尽责性个体的个性化广告匹配效果并不稳定,但其偏好词语却展现出高开放性的语言特征。这表明,不同人格特质可能并非完全独立,而是相互影响,并在个性化广告偏好中表现出复杂的交互关系。因此,未来研究可以进一步探讨多个特质的组合如何影响个性化广告的接受度。例如,开放性与尽责性的交互是否会影响信息复杂度的接受度?外倾性与神经质的组合是否会影响情绪导向广告的有效性?这些问题值得在后续研究中进一步挖掘。此外,个性化广告的匹配策略通常基于单一特质的优化,但如果能够将多个特质融合进 AI 的个性化生成机制中,可能会更精准地满足个体的真实需求。未来研究可以尝试构建多维特质的匹配模型,探索如何在 AI 生成广告时,综合考虑个体在多个特质上的表现,以优化广告的说服力和个性化效果。

其次,特征选择上,目前,研究主要依赖于大五人格特质作为个性化广告的匹配维度,而在实际应用中,受众的个性化信息不仅来自于自我报告的人格测量,还可以通过其他行为数据进行建模。例如,社交媒体上的互动模式、历史购物记录、兴趣标签等都可以反映个体的个性特征,甚至在某些情况下,比传统人格测量更具预测力。已有研究尝试利用大五人格特质预测个体的消费偏好和社交媒体行为\citep{kosinski2013private, golbeck2011predicting},那么,未来研究可以进一步探索,AI 是否可以直接基于这些数据进行特征提取和个性化广告生成,而不依赖于传统的人格测量?这不仅可以降低用户数据收集的成本,也可能提升个性化广告的适用性和精准度。此外,AI 生成广告的个性化能力是否可以进一步结合受众的实时行为反馈进行动态优化?例如,AI 是否可以基于用户的实时交互(如点击、浏览时长、互动行为等),持续调整广告的个性化程度,以提高广告的效果?这些问题都值得未来研究深入探讨。

第三,广告形式上,本研究主要关注 AI 在文本广告创作中的个性化能力,而在实际营销环境中,广告通常包含更丰富的形式,如图文广告、短视频广告、甚至虚拟现实(VR)广告等。已有研究表明,视觉元素在个性化广告传播中发挥着至关重要的作用,消费者的情绪反应、认知加工和记忆效果可能受到广告图像、色彩、动态效果等多种因素的影响\citep{matz2019predicting,segalin2017your}。然而,AI 在视觉内容生成方面的能力相较于文本仍然处于发展阶段,因此未来研究可以进一步探索 AI 生成的个性化广告在不同媒介上的适用性。例如,AI 生成的图像或视频广告是否能够针对个体的人格特质进行优化?不同人格特质个体是否对不同类型的视觉元素存在偏好?此外,AI 生成的多模态广告内容(如文本+图片+视频)\citep{zhang2024mm} 是否比单一文本广告更具说服力?这些问题都可以成为未来研究的重要方向。

第四,关于因变量的选择,本研究虽然在实验设计中选择了相对贴近现实的衡量指标,如广告态度、购买意愿和社交媒体的互动意愿,但仍然无法完全替代真实市场环境下的消费者行为。例如,尽管个体可能在实验中表达对某一广告的较高偏好,但这种偏好是否能够转化为实际的购买行为仍然存疑。在实际市场环境中,个性化广告的最终目标是促进消费者的购买决策,因此未来研究可以进一步引入真实的市场数据,如实际购买转化率、广告点击率、用户长期互动行为等,以更全面地评估 AI 生成个性化广告的实际效果。此外,AI 生成广告的长期影响尚未被充分考察。本研究主要关注 AI 生成广告的即时效果,而在现实环境中,广告的累积效应可能会影响受众的态度变化。例如,受众在长期接触 AI 生成的个性化广告后,是否会产生信任疲劳或适应性反应?未来研究可以采用纵向研究方法,跟踪 AI 生成广告的长期影响,以更全面地评估其实际应用效果。

最后,理论模型的构建上,本研究主要基于个性化广告、信息加工与感知相似性的理论框架来解释 AI 生成广告的有效性,但仍有许多相关理论可以进一步补充。例如,受众对 AI 生成广告的信任度可能受到技术接受模型的影响\citep{davis1989perceived},即个体对 AI 生成广告的接受度可能与其对 AI 技术的感知易用性和感知有用性相关。此外,广告的情感共鸣机制可以结合人际沟通理论进行更深入的探讨,特别是在 AI 生成的宜人性广告中,人类创作者的表达方式为何比 AI 更具影响力?此外,随着 AI 生成模型的不断发展,未来研究可以结合计算社会科学的方法,探索 AI 生成个性化广告如何通过大规模数据分析优化个性化匹配机制,并基于更复杂的个性化模型进行精准传播。

综上所述,尽管本研究在 AI 生成个性化广告的有效性和适用性上提供了新的见解,但仍然存在多个值得进一步研究的方向。未来研究可以在个性化匹配的复杂性、数据特征的多样性、广告形式的拓展、因变量的现实适用性以及理论基础的深化等方面进行更深入的探索。这不仅能够拓展 AI 在个性化广告中的应用范围,也将为个性化广告的理论研究提供更系统的支撑。