\chapter{引言}
在当今数字化时代,广告行业经历了从传统广播式传播向精准个性化传播的深刻转变。传统广告模式(如电视、报纸和杂志广告)通常采用大规模、非定向的投放策略,即相同的广告内容面向所有观众,而不考虑个体特征的差异。这种模式虽然能够覆盖广泛的受众,但由于缺乏针对性,广告的相关性较低,容易被忽视或引起受众反感。例如,电视广告通常按照固定时段播放,无论观众的年龄、兴趣或需求如何,都接收到相同的信息,这导致部分受众对广告内容缺乏兴趣,从而削弱广告的传播效果。随着互联网和数据分析技术的进步,广告行业逐步迈向个性化广告,通过精准的用户画像和数据分析,使广告内容更具针对性,从而提升广告的有效性 \citep{teeny2021review}。

个性化广告的核心在于利用用户的行为数据(如浏览历史、搜索记录、购买行为等)来匹配广告内容,使其更符合用户的兴趣和需求。借助大数据分析和机器学习技术,广告商能够预测用户可能感兴趣的产品或服务,并基于此进行精准投放。例如,社交媒体平台和电商网站已经广泛应用个性化推荐系统,根据用户的过往互动数据推送相关广告,使广告与用户需求的匹配度显著提升。然而,基于短期行为数据的个性化推荐存在一定局限性,用户的兴趣和需求可能会随着时间发生变化,导致推荐的精准度下降。因此,研究者开始关注更稳定的个体特征,如人格特质,在个性化广告中的应用 \citep{backteman1981longitudinal}。人格特质作为个体长期稳定的心理特征,相较于短期兴趣或行为数据,能够更持久地预测个体的偏好、决策风格和广告接受度。近年来,研究表明,通过分析社交媒体上的语言风格、互动模式和消费行为,可以较为精准地预测个体的大五人格特质 \citep{markovikj2013mining}。这一技术进步为广告个性化提供了新的可能性,使广告不仅能基于用户的短期行为进行推荐,还可以基于其长期稳定的心理特征进行定制,从而提高广告的持久影响力。

已有研究表明,基于人格的个性化广告具有较高的可行性和有效性。例如,\citet{matz2017psychological} 通过在 Facebook 进行的大规模广告投放实验,验证了基于大五人格的广告定制策略的效果。研究发现,针对不同人格特质量身定制的广告相较于通用广告,能够显著提高广告的点击率和购买转化率。例如,针对高开放性或高尽责性个体设计的广告内容在吸引注意力和促进购买行为方面表现更优。这一效果的核心机制在于个性化广告能够增强受众对广告的心理契合感\citep{teeny2021review},即当广告内容与个体人格特质相匹配时,受众会更容易产生共鸣,并对广告持更积极的态度。

综上所述,基于大五人格理论的个性化广告已在理论与实践层面展现出显著价值。然而,传统的人格化广告在内容创作效率与文案多样性上仍面临诸多限制。随着生成式AI(Generative AI)技术的飞速发展,尤其是以GPT为代表的大规模语言模型在自然语言处理领域取得的重大进展,为个性化广告的创作带来了全新的技术动能。一方面,GPT在文本理解与生成上具有高效且灵活的特征,能够根据不同人格维度快速生成大量多样化的广告文案 \citep{matz2024potential},帮助我们更好地理解人格特质的需求;另一方面,生成式AI也引发了新的研究议题,例如AI生成的文案是否真正切中不同人格特质的核心需求,以及在信息披露情境下,AI生成广告的可信度与说服力会受到何种影响。因此,本研究将结合大五人格理论与生成式AI技术,综合运用实验方法、文本分析和机器学习等手段,对AI在个性化广告中的可行性与有效性展开系统评估,并进一步探讨AI与人类专家在创意与说服机制上的差异与协同潜力。通过这一研究,不仅能够深化我们对人格化广告的学术理解,也将为广告行业在数字化转型背景下借助生成式AI提升个性化广告的效能提供重要的实证支持与实践启示。



\section{个性化广告}
\subsection{个性化广告的定义}
个性化广告是指根据个体特征、兴趣偏好或行为数据,对广告内容进行定制化调整,以提升广告的相关性和说服效果 \citep{dijkstra2012personalization}。其核心在于在信息内容中融入受众可识别的个性化特征,从而提高广告的注意力吸引力和信息处理深度。根据个性化的实现方式,研究者将其区分为线索式个性化(cue-based personalization)和特质式个性化(trait-based personalization)\citep{winter2021effects}。

线索式个性化主要依赖显性个人信息,即广告中直接呈现用户的姓名、地理位置、过往购买记录等个体化信息。例如,在电子邮件营销中\citep{maslowska2016all},研究发现,邮件标题中包含用户姓名(如“Anne,这款新品可能正合你意!”)能够显著提高邮件的打开率和点击率。主要依赖于在广告中加入受众的个人信息(如姓名、公司、所在地等)来增强广告的关注度。在数字广告中,某些品牌会根据用户的地理位置定制广告内容,如外卖平台向特定城市用户推送“您附近3公里内的热门餐厅”广告 \citep{lambrecht2013does}。此外,线上购物网站常利用用户的浏览和购买记录,生成基于历史行为的推荐广告,例如亚马逊的“猜你喜欢”广告系统,这种策略能够有效提升转化率。眼动追踪研究进一步表明,受众在观看个性化广告时,尽管未必会立即点击广告,但会对其投入更多的注视时间 \citep{pfiffelmann2020personalized}。然而,尽管线索式个性化能够有效吸引注意力,其效果并非始终正向。一些研究发现,过度依赖个人信息的广告可能引发心理抗拒或隐私顾虑,进而降低受众对广告的接受度 \citep{chen2019understanding}。这一“个性化—隐私悖论”表明,当消费者意识到自己的个人信息被利用时,可能会对广告产生抵触情绪,影响广告的说服力 \citep{awad2006personalization}。

相较而言,特质式个性化(trait-based personalization)并不依赖于显性个人信息,而是基于受众的人格特质或心理属性对广告内容进行深度定制。例如,个性化广告可以通过调整信息框架、情感基调或广告叙事方式,使广告内容更符合特定人格特质群体的偏好 \citep{matz2017psychological,hirsh2012personalized}。这种个性化方式的优势在于,它避免了线索式个性化可能引发的隐私侵犯,同时能够更深层次地影响个体的态度与行为。例如,\citet{matz2017psychological} 研究发现,针对外向性强的用户,广告可以采用更加生动、社交互动导向的语言(如“加入我们,一起探索新世界!”),而针对内向性用户的广告则更偏向个人化和安静体验(如“享受属于你的独处时光”)。此外,除了大五人格,其他心理特质也被用于特质式个性化广告设计。例如,基于调节定向理论,广告可以针对“促进导向”用户强调机会与成长(如“开启你的成功之路”),而针对“预防导向”用户则强调安全性与责任(如“确保万无一失”)\citep{cesario2008regulatory}。综上所述,个性化策略的类型有多种,并会进一步影响受众信息处理方式的影响。线索式个性化依赖显性个人信息来吸引注意力,但可能引发隐私顾虑,而特质式个性化通过与人格特质匹配的方式影响受众的深层认知过程,在增强广告说服力的同时,减少隐私侵犯的风险。

\subsection{个性化广告的效果衡量}

近年来,个性化广告的效果成为广告与营销研究的重要议题,相关研究主要围绕实验室研究与真实市场研究两种方法展开。这些研究不仅关注实验设计如何操控个性化程度,还考察不同研究场景下的个性化广告效果,以及影响广告有效性的关键变量。

在\textbf{实验设计}方面,研究者通常通过操控个性化程度来评估其影响,主要方式包括对比个性化广告与非个性化广告、操控个性化信息的匹配度,以及引入错配条件等。首先,在个性化广告与中性或非个性化广告的比较中,研究者主要探讨个性化信息本身是否能够提升广告的效果。例如,\citet{de2015me} 研究了基于性别的个性化广告,其中个性化广告直接提及受众的性别,如“这款产品是专为有自信的\textbf{男性/女性}设计的”,而非个性化广告则采用性别中立的表述,如“这款产品是专为有信心的消费者设计的”。类似地,\citet{ho2008personalization} 研究了个性化推荐系统的有效性,实验条件包括高匹配广告(基于用户的历史偏好生成)和低匹配广告(随机推荐),以评估个性化推荐是否能提高消费者的购买意愿。研究结果普遍表明,相较于非个性化广告,个性化广告能够增强消费者的广告关注度,提高购买意愿和品牌认同感。其次,在操控个性化匹配度的实验设计中,研究者关注广告内容与个体特征的契合程度对广告效果的影响。例如,\citet{aguirre2015unraveling} 通过操控个性化信息的明显程度,设置无个性化(不包含任何个人信息)、中等个性化(基于用户行为数据)和高度个性化(结合用户行为数据与人口统计信息)等不同级别,以探讨个性化程度对受众接受度的影响。研究发现,在个性化程度适中的情况下,受众对广告的接受度最高,而高度个性化可能引发隐私担忧,降低广告的整体有效性。最后,在匹配与错配的对比实验*,研究者探讨当广告内容不符合个体特征时是否会影响其接受度。例如,\citet{moon2002personalization} 研究了消费者的人格特质与广告个性化匹配的关系,实验根据受试者在人际交往中的主动性或被动性特质调整广告信息,结果发现,当个性化信息与受众特质完全错配时,消费者的广告接受度显著下降,甚至可能产生负面反应。综上所述,实验研究揭示了个性化广告的有效性受到个性化程度、信息匹配度以及隐私感知等因素的综合影响,合理的个性化策略有助于提高广告的传播效果,而过度个性化或错配则可能引发受众的抵触情绪,削弱广告的说服力。

\textbf{在研究场景上},个性化广告的效果主要通过实验室研究和真实市场研究进行测量。实验室研究通常采用严格控制的实验设计,以消除外部干扰因素,并通过问卷调查或行为观察测量受试者的反应。此类研究多以大学生为被试,主要考察受众对广告或品牌的态度、购买意愿等主观指标,以及模拟任务中的点击行为。例如,\citet{hirsh2012personalized} 发现,当广告内容的语调和信息传递方式与目标受众的人格特质相匹配时,受众对广告的态度更加积极,并表现出更高的购买意愿。然而,实验室环境的人工性可能导致研究结果的外部效度受限,即受试者在实验室中的反应未必能完全反映其在真实消费环境中的决策过程。相比之下,真实市场研究通常采用现场实验或大规模数据分析,以评估个性化广告在自然情境中的实际效果。例如,\citet{matz2017psychological} 通过在 Facebook 进行的大规模现场实验,测试基于用户心理特征定制的广告与不匹配广告的表现,结果显示,匹配受众人格的广告点击率提高了 40\%,购买转化率提高了50\%。此外,\citet{bhavsar2024trending} 采用问卷调查分析个性化广告对不同年龄群体的影响,发现 64.1\% 的消费者对个性化广告持正面态度,33.3\% 的消费者因个性化广告而将商品加入购物车,40.8\% 对个性化广告的整体体验持积极评价。这些研究表明,在实际市场中,个性化广告不仅能够提高用户的品牌参与度,还可能促进购买决策的转化。

在衡量个性化广告的效果指标方面,研究通常采用行为反应和心理反应两类指标。行为反应包括购买意愿、点击率和转化率,尤其在真实市场环境中,点击率和转化率是衡量个性化广告效果的关键因素。例如,\citet{matz2017psychological} 发现,与不匹配广告相比,匹配广告的点击率和购买转化率显著提高。此外,\citet{lambrecht2013does} 研究了动态再定向广告(即反复向消费者推送其曾浏览过的商品),结果表明,这类高度个性化的广告在整体上并不比普通品牌广告更有效,甚至在某些情况下会引发受众的反感,导致广告效果下降。心理反应主要包括广告态度、品牌态度以及信息记忆等。例如,\citet{moon2002personalization} 研究发现,当广告风格与消费者个性匹配时,受众对广告的态度更积极,并更愿意接受广告推荐的产品。\citet{de2015me} 进一步发现,个性化广告尤其在品牌知名度较低时能有效增强消费者的品牌态度。此外,\citet{bang2016tracking} 通过眼动追踪实验发现,个性化广告相比于标准化广告,能让受众更频繁地注视广告内容,并花费更长的时间。然而,\citet{maslowska2016all} 指出,尽管个性化广告可以增强信息记忆,但如果个性化信息过于明显,可能激活消费者的抗拒心理,反而削弱广告的说服力。

总体而言,大量跨情境研究表明,个性化广告相较于非个性化广告通常能显著提升广告的效果,尤其是在提升受众参与度和品牌态度方面。例如,\citet{hirsh2012personalized} 的研究证实,基于个性化内容的广告相比通用广告能有效促进消费者的购买意愿。然而,个性化广告的具体效果因研究设计、个性化类型和应用场景的不同而有所变化。例如,\citet{winter2021effects} 发现,个性化广告仅在针对特定的说服易感性(如权威影响)时,才会显著提升用户的互动意愿,而在其他情境下,个性化信息并未稳定改善消费者对广告的态度。\citet{aguirre2015unraveling} 研究发现,当信息收集方式不透明时,个性化程度越高,消费者的点击率反而显著下降,说明隐私担忧可能抵消个性化广告的积极效果。此外,\citet{lambrecht2013does} 研究了动态再定向广告(即反复向消费者推送其曾浏览过的商品),结果发现,这类高度个性化的广告在整体上并不比普通品牌广告更有效,甚至在某些情况下会引发受众的反感,导致广告效果下降。综上所述,个性化广告在多数情况下能有效提升广告的吸引力和转化率,但其效果受到多个因素的影响,有关个性化广告的心理机制及其影响因素将在后续章节进一步探讨,以更全面地理解个性化广告如何影响消费者的态度与决策过程。(详见\ref{个性化说服:理论与机制})。

\section{基于大五人格的个性化广告}

\subsection{大五人格模型概述}

在人格心理学领域,研究者普遍认为个体的人格并非单一维度所能概括,而是由多个相对独立但彼此关联的特质构成。因此,为了更全面地理解人格特征并建立稳定可测的人格结构模型,心理学家提出了多种人格分类体系。这些模型不仅用于解释个体的行为模式,也广泛应用于市场营销、消费者行为研究和个性化推荐系统等实际领域。

现有的人格分类模型各具特色,主要包括艾森克人格模型(Eysenck’s Personality Model)、MBTI(Myers-Briggs Type Indicator)、HEXACO模型和暗黑三角(Dark Triad)模型。艾森克人格模型 \citep{eysenck2017biological} 主要基于神经科学视角,将人格归为外倾性、神经质和精神质三个维度,并强调个体的生理基础差异。MBTI人格类型指标 \citep{briggs1976myers} 以二分法将人格划分为16种类型,在职业规划与团队管理中较为常见,但由于其测量稳定性较低,学术界对此模型的认可度较低 \citep{pittenger2005cautionary}。HEXACO模型 \citep{ashton2007empirical} 在大五人格的基础上新增了“诚实-谦逊”维度,用于研究道德行为和利他性特征。暗黑三角模型 \citep{paulhus2002dark} 则专注于个体的“黑暗人格”特质(如马基雅维利主义、心理病态和自恋)及其对社会行为的影响。这些模型虽各具优势,但在广泛的心理学研究和实际应用中,大五人格因其理论基础扎实、跨文化适用性强、测量稳定性高,而成为当代人格研究的主流框架。

大五人格理论基于词汇学假设和因素分析方法,将个体的人格特质归纳为五个核心维度:开放性、尽责性、外倾性、宜人性和神经质 \citep{mccrae1992introduction}。现已被广泛应用于社会心理学、消费行为学和市场营销研究等多个领域中。每个维度均衡量个体在特定人格特征上的强弱程度,并与个体的认知、情绪和行为模式密切相关。例如,开放性反映了个体对新体验、创意和抽象概念的接受程度;尽责性反映了个体的自律性、责任感和计划性;外倾性代表了个体的社交活跃度和对外部刺激的偏好;宜人性体现了个体在社交关系中的合作性和共情能力;神经质则衡量了个体的情绪稳定性和对压力的敏感度 \citep{costa1995domains}。大五人格模型之所以成为当代人格心理学的主流框架,主要得益于其在不同文化、语言和种族群体中的一致性。大量跨文化研究表明,该模型在不同社会背景下均能稳定地刻画个体的人格特征。例如,\citet{mccrae2002nature} 在对50多个国家进行的大规模研究中发现,大五人格的五个维度在全球范围内均能被验证,且不同文化群体对这些特质的理解较为一致。这种跨文化稳定性使得大五人格成为人格研究中最具普适性的理论框架之一。此外,大五人格在纵向研究中的稳定性也得到了实证支持。例如,\citet{caspi2005personality} 通过长达20年的追踪研究发现,尽管个体的人格特质可能受到教育、职业和社交经历的影响而略有变化,但其核心特质仍然保持较高的稳定性。

综上所述,大五人格因其理论扎实性、测量稳定性和跨文化一致性,成为人格心理学中最具代表性的人格分类框架,并在社会科学、商业决策、市场营销、消费者行为研究等领域得到了广泛应用。其高度稳定的特征,使其成为研究个性化广告、消费者行为预测和个性化推荐系统的重要理论依据,为个性化营销提供了科学支持。

\subsection{大五人格与说服偏好}

研究表明,大五人格不仅影响个体的行为和决策模式,也决定了他们在接受说服信息时的敏感性和倾向性。不同人格特质的个体在信息处理方式、认知风格以及对劝服策略的接受度上存在系统性差异,这些差异影响着个体在面对不同类型信息时的态度和反应 \citep[例如][]{alkics2015impact}。

\textbf{开放性}高的个体通常具有较强的求知欲,乐于接受新事物,倾向于探索创新性和抽象概念。在说服策略方面,研究表明,高开放性个体对社会压力和工具性利益相关的劝导策略更为敏感 \citep{gerber2013big},即他们更容易受到强调社会责任和实用价值的信息影响。此外,低开放性个体更容易受到权威和社会共识策略的影响 \citep{oyibo2017investigation},意味着他们更倾向于遵循专家意见或从众行为。\textbf{尽责性}高的个体倾向于计划性强、自律,并重视承诺和责任感。研究表明,他们通常对承诺和互惠策略更敏感,但较不容易受到喜好策略的影响 \citep{oyibo2017investigation}。这表明,他们更倾向于基于长期价值、责任感作出决策,而不是依赖外部情感因素。\textbf{外倾性}水平高的个体通常具有较强的社交倾向,乐于接受群体互动和积极情绪体验。他们对社交性强的广告、互动式营销策略更为敏感,但对社会压力型信息的抵抗力较强。研究表明,高外倾性个体比低外倾个体更容易受到社会压力策略的影响,这可能是因为低外向性个体更在意社会认同感和群体归属感。此外,研究发现,高外倾的个体更容易受到稀缺性策略的影响,即当产品被宣传为限量供应或独特机会时,他们更可能产生购买冲动 \citep{wall2019personality}。\textbf{宜人性}高的个体倾向于合作、信任他人,并重视社会规范和道德价值观。他们对权威、承诺和喜好策略更为敏感 \citep{oyibo2017investigation},但对强调个人主义和竞争性的广告内容反应较差 \citep{gerber2013big}。\textbf{神经质}高的个体通常情绪波动较大,容易受到外部环境的影响,且对不确定性和风险较为敏感。研究发现,他们更容易受到社会共识策略的影响,即当广告强调“大家都在使用”或“这是市场主流选择”时,他们更可能做出购买决策 \citep{wall2019personality}。此外,神经质高的个体可能对强调安全感、稳定性和情绪安抚的广告更有反应。

综上所述,大五人格特质影响个体在面对不同说服策略时的接受度。开放性高的个体倾向于接受强调创新、社会责任或实用价值的劝导,而低开放性个体则更容易受到权威和社会共识的影响。尽责性个体更容易受承诺和互惠策略的影响,而外倾性个体更倾向于稀缺性和互动性较强的信息。宜人性个体则对权威、社会认同和道德责任相关的策略更敏感,而神经质高的个体则更容易受到社会共识和安全感导向的信息影响。这些特质差异不仅影响个体对信息的处理方式,也决定了他们在不同情境下的说服接受模式。


\subsection{基于大五人格的个性化广告效果}

基于前述研究可见,大五人格在信息接受方式、广告内容偏好以及个体决策机制上均存在显著差异。正因如此,大五人格已成为个性化广告设计的重要理论基础,能够帮助广告商更精准地调整广告策略,以匹配不同受众的个性特质,提升广告的个性化效果。个性化广告的核心在于通过调整信息框架、措辞、情感表达及视觉呈现,使广告内容与受众的人格特质相匹配,从而增强广告的个性化感知,提高说服力。已有研究表明,与非个性化广告相比,基于大五人格定制的个性化广告能够显著提升广告的吸引力、信息加工深度和品牌认同感。

\citet{hirsh2012personalized}针对一款手机产品,设计了分别对应大五人格五个维度高水平的五种不同广告。例如,外倾性个性化广告(针对高外倾的个体)强调“你将永远置身刺激精彩之中”,尽责性个性化广告(针对高尽责的个体)则突出“这将简化你的工作生活,提高效率”。实验采用重复测量设计,每位被试依次观看五种广告,并对广告的吸引力和说服力进行评分。研究者通过多重回归分析,以被试的大五人格得分预测其对各广告版本的评价。结果表明,个体对与自身人格特质相匹配的广告评价更高,而其他人格维度的得分对该广告的评分无显著影响。这一发现表明,个性化广告的效果主要取决于广告内容与受众人格的匹配,而非个体整体的广告态度,为人格驱动的广告定制策略提供了实证支持。

\citet{matz2017psychological}进一步在真实市场环境中检验了人格匹配广告的有效性。他们在 Facebook 上进行了一项大规模广告投放实验,利用用户的数字足迹预测其外向性和开放性水平,并基于这些特质设计了相应的广告内容,即外倾性 vs. 内倾性广告,以及高开放性 vs. 低开放性广告。例如,外倾性广告采用充满社交活力的语调,如“跳舞吧,就算没人看见”,而内倾性广告则更含蓄,如“美丽无需高调张扬”。实验采用 2×2 被试间设计,由于是在真实市场环境中进行的实地实验,每名消费者仅会接收到一则广告。研究者采用分层逻辑回归分析,以点击(点击=1,未点击=0)和转化(转化=1,未转化=0)作为因变量,预测变量包括消费者人格(高/低外向性或开放性)、广告人格(外倾性/内倾性或高开放/低开放)及二者的交互项。分析结果表明,人格匹配广告显著提升广告的点击率和转化率。具体而言,与不匹配广告相比,匹配广告的点击率提高约40\%,购买转化率提高近 50\%。这一研究不仅验证了人格定向广告在真实市场中的有效性,还表明即使是低水平特质人群(如低外倾消费者),也能从个性化广告中受益,进一步支持了基于人格的广告定向策略的商业应用价值。

\citet{winter2021effects}采用在线实验进一步考察人格匹配广告在不同情境下的效果。他们在实验中操控广告的信息框架是否与受众的人格特质相匹配,设计了 9 种广告版本,其中 5 种基于大五人格(外向性、宜人性、尽责性、神经质、开放性,均针对高水平设计),4 种基于其他心理特质(如权威影响、从众效应等)。被试被随机分配至不同广告条件,每人仅观看一则广告,并在观看后填写问卷,评价广告态度和购买意向。研究采用后测分组设计,即被试观看广告后再测量其人格评分,并根据人格得分判断广告是否匹配。结果发现,相比于不匹配广告,匹配广告在广告态度、购买意向和社交参与意向上均有小幅提升。然而,进一步分析发现,匹配效应在不同人格特质上的表现并不均衡,例如,针对权威型、从众型等心理特质的匹配效果更为显著,而大五人格匹配的影响则相对较弱。这一研究表明,人格匹配广告的效果可能受到广告类型和产品情境的调节,并非所有人格匹配策略都能稳定提升广告效果。

在实际营销应用中,Hilton酒店集团与剑桥大学心理测量中心合作进行了一项Facebook广告投放实验,利用大五人格预测目标客户的人格特质,并基于此定制了10种不同风格的广告。实验结果表明,匹配广告的点击率(CTR)至少是行业平均水平的两倍,社交媒体分享率也显著提高。具体而言,所有人格匹配广告的CTR均远高于旅游行业在Facebook广告上的基准值(0.08\%)。即使是表现最差的目标群体(宜人性匹配广告),其CTR也达到了0.17\%,仍然比行业基准高出200\%。外向性匹配广告的表现最佳,其CTR为0.27\%,比行业基准高出340\%。这一案例说明,人格定向广告不仅适用于实验研究,也能在真实市场环境中有效提升广告效果。

综合上述研究,可以得出以下三个关键发现。首先,实验材料的有限性是个性化广告研究的普遍问题。大多数研究中,每个个性化条件下通常仅生成一则广告,而广告的制作过程较为复杂。这种单一化的广告设计限制了研究对个性化策略的全面测试,可能影响结果的稳定性,并受到单一实验材料的随机影响。其次,研究主要围绕高水平特质受众进行定制,而对低水平特质的关注较少。尽管少量研究涉及低水平特质,但往往仅选择个别特质进行测试 \citep[如][]{matz2017psychological}。例如,尽责性、宜人性、神经质等低水平特质个体通常未被纳入个性化广告设计,而仅作为对照组接受高水平特质广告。这种设计限制了研究对低水平特质用户行为反应的理解,也难以评估是否所有人格维度的个性化广告都同样有效。最后,不同人格特质的个性化广告效果存在差异,部分特质的匹配效应更稳定,而其他特质的匹配效应较弱或不确定。现有研究较一致地发现,外向性和开放性匹配广告的效果最为显著,而尽责性、宜人性和神经质匹配的广告效果相对不稳定。因此,未来研究需要结合更多元的广告材料和低水平特质个体,进一步探索大五人格个性化广告的适用范围及最佳匹配策略。


\section{个性化说服:理论与影响因素}
\label{个性化说服:理论与机制}
\subsection{个性化说服的理论基础及心理机制}
\label{个性化说服理论}
个性化广告的说服效果并非由单一机制解释,而是由多个心理机制共同作用的结果。其中,精细加工可能性模型(Elaboration Likelihood Model,ELM)作为个性化说服的核心理论框架 \citep{petty1986elaboration},为理解个性化广告的有效性提供了理论基础。ELM提出两种基本的说服路径:中枢路径和外周路径。个体通过中枢路径深入加工信息,形成稳定而持久的态度改变,而外周路径则更多依靠简单的表面线索。大量实验证据显示,个性化广告能提高受众感知到的信息个人相关性,有效提升信息加工的动机和投入程度,使受众更可能通过中枢路径深度加工信息,从而带来更持久、更稳固的说服效果 \citep{tam2005web, hirsh2012personalized}。例如 \citet{tam2005web}发现,当广告内容与受众自身兴趣或偏好紧密相关时,受众的认知投入程度显著提高,表现出更强的购买意愿以及品牌偏好度。

进一步而言,“个人相关性”这一关键因素可通过两个具体的心理机制展开,即社会认同理论(Social Identity Theory)与自我参照理论(Self-Referencing Theory)。社会认同理论提出,个体的自我概念部分来源于其所属的社会群体 \citep{rogers1977self}。当个性化广告以受众所属群体的身份或价值观作为切入点时,更容易引发个体对群体身份的认同感,使广告内容与受众的自我概念密切相关,从而提升信息加工动机和态度变化。例如,研究发现,当广告直接唤起消费者的群体归属感(如特定品牌忠诚顾客、特定兴趣群体成员)时,消费者对广告的关注程度和参与意愿显著提高 \citep{mols2012makes}。在一项实验中,当消费者看到以自己所属群体(例如“运动爱好者”、“环保倡导者”)为定位的个性化广告时,他们表现出更高的品牌偏好和消费意愿,相比未使用群体身份诉求的通用广告显著提高了购买意愿和品牌态度 \citep{white2013and}。

而自我参照理论强调,个体更倾向于加工和记忆与自身相关的信息 \citep{rogers1977self}。当个性化广告通过明确的个人兴趣、行为历史或人口统计特征唤起个体的自我参照过程时,受众更容易感受到广告信息与自己的契合度,从而提升对广告内容的精细加工与记忆效果 \citep{burnkrant1989self}。在实验研究中,相比通用广告,含有明确自我参照信息(如名字、过往偏好或个人兴趣)的个性化广告显著提升了消费者的信息记忆与购买倾向 \citep{summers2016audience}。即使在受众认知资源或动机较低时,自我参照也可以作为一种强有力的启发式线索,提高对广告的直觉性偏好。这种作用方式也体现了ELM中的外周路径机制。

整体来看,社会认同理论和自我参照理论分别在群体层面和个体层面深化了ELM模型中个人相关性的概念,提供了理解个性化广告如何促进信息加工深度并提升广告效果的具体机制。这种融合理论视角为解释不同情境下个性化广告效果的差异提供了坚实的理论基础。


\subsection{个性化说服效果的影响因素}
个性化广告的有效性受到多种因素的影响,其中既包括影响个性化广告作用机制的中介变量,也包括调节个性化广告效果的个体特征和情境因素。基于前文\ref{个性化说服理论}的理论框架,已有研究进一步验证了这些机制,并探讨了影响个性化广告效果的关键因素。整体来看,相似性感知是个性化广告影响消费者态度和行为的核心中介机制,而产品类型和消费者人格特质则会调节这种作用。此外,隐私担忧、认知需求等因素可能进一步影响个性化广告的说服效果,甚至在某些情况下削弱或逆转其正面作用。以下将分别探讨这些关键变量的作用机制。

在个性化广告中,消费者对感知个性化、感知相关性和感知相似性的主观评价,显著影响他们对广告的态度和购买意愿。已有实验研究操控社交媒体广告的个性化程度,发现用户觉得广告“为我量身定做”(感知个性化)时,会认为广告内容更切合自己需求(感知相关性更高),从而对品牌态度更积极、点击购买意愿更强 \citep{de2015me}。不仅感知相关性,感知相似性也被发现是个性化广告影响消费者反应的关键心理机制。例如,在针对特定身份群体的定向广告中,当消费者认为广告中的形象或措辞与自己“很像”时,会产生更高的认同感,进而提升广告态度和购买意向\citep{li2016does}。不仅如此,相似性感知的影响在不同情境下可能有所不同。已有研究发现,在某些类别的产品(如娱乐性产品)中,相似性感知在消费者认同与广告态度间起到了部分中介作用 \citep{madadi2021impact}。例如,种族或性别匹配的广告模型已被证明对广告态度有正向影响,尤其在涉及个体身份认同的广告策略中,相似性扮演了至关重要的角色。综上,感知个性化、感知相关性和感知相似性均能够在个性化广告与消费者响应之间发挥中介作用,但其具体效果仍受到个体特征和情境因素的影响。

然而,相似性感知的积极作用并非在所有情况下都能稳定发挥作用,不同消费者的个性特质可能会影响他们对个性化广告的接受度。过往研究发现,不同人格的个性化广告效果存在显著差异。例如,尽责性和神经质的个性化广告效果并不总是稳定 \citep{matz2024potential},而基于说服易感性策略构建的个性化广告往往更有效 ,基于说服易感性策略构建的个性化效果更好 \citep{winter2021effects}。研究表明,尽责性高的个体通常更注重计划和隐私,因此更容易对个性化广告产生反感,甚至表现出更高的广告回避倾向 \citep{cao2024effects}。相反,亲和性较高的个体更倾向于信任广告主,因此对个性化广告的接受度更高,广告回避倾向较低。除了个体特征,产品类型也是影响个性化广告效果的重要因素。虽然关于产品类型的调节作用目前仍缺乏系统性研究,但已有文献提供了一些间接证据,表明功利型产品和享乐型产品在个性化广告中的表现可能存在差异。实用型产品注重满足功能需求,消费者更看重信息的契合度;享乐型产品则侧重情感体验和愉悦感。因此,针对不同产品类型,个性化策略效果可能不同。有研究 \citep{tsekouras2024don}表明如果明确告诉用户广告是根据其数据定向的(高显性个性化),总体效果会下降,推测是激发了隐私顾虑;但有趣的是,此负面影响在功利型信息框架下更明显,而在享乐型信息框架下有所缓解。也就是说,当广告诉求偏功利理性时,消费者更容易留意到数据被利用从而反感;而享乐诉求因情感愉悦掩盖了部分隐私顾虑,能缓冲个性化的负面效应。

除了个体特征和产品属性,隐私担忧也是影响个性化广告效果的重要因素。尽管个性化广告可以提高消费者的相似性感知,但如果消费者对数据隐私存在顾虑,这种正面作用可能被削弱甚至逆转。已有研究表明,高隐私敏感度的消费者在面对高度个性化的广告时,可能会产生更强的心理抗拒,认为广告商正在“监视”他们,从而降低广告的接受度 \citep{lina2021privacy}。此外,认知需求也可能影响消费者对个性化广告的反应。高认知需求个体更倾向于深度处理信息,因此如果个性化广告提供了高质量的论据,他们可能会更积极地接受广告内容。而低认知需求个体则更依赖启发式判断,他们可能会更容易受到表面化的个性化线索(如名字、推荐语)的影响。这表明,不同的消费者群体可能需要不同的个性化广告策略,以优化广告的说服效果。

\section{AI在个性化广告中的应用}
随着生成式人工智能的发展,AI能够自动生成广告文案、新闻报道和社交媒体帖子等文本内容,并以此影响受众的态度和行为。研究表明,大型语言模型生成的劝服信息在许多领域(政治、营销、公共健康、电商、慈善等)已经达到与人类相当甚至更高的说服力。这种高度自动化、个性化的内容生成为传播带来了新的机遇和挑战。本节将从AI生成文本的说服效果、AI在个性化广告中的具体应用、AI作为信息源对受众的影响三个方面展开深入论述,并明确未来的研究方向。

\subsection{AI生成信息原理}

近年来,生成式人工智能(Generative AI),尤其是大语言模型(Large Language Models, LLMs),在文本生成与自动化内容创作领域取得了显著突破。与传统基于规则或分类的人工智能不同,生成式AI不仅能够分析和筛选已有信息,还能够自主生成符合语境、逻辑连贯且富有说服力的文本。这种能力使其在营销、政治传播、公共健康和新闻报道等多个领域的说服性信息生成方面展现出巨大潜力。

生成式大语言模型的核心原理基于大规模语料库训练和上下文驱动的文本预测。在训练阶段,这类模型通过学习海量的自然语言数据(如新闻、社论、学术论文、广告文案、社交媒体帖子等),识别语言模式、句法结构和语义关系,并归纳出符合自然语言表达的规律。在实际应用中,AI根据给定的输入(如问题、主题或指令),逐步预测最可能出现的单词或短语,以构建连贯且具有逻辑性的文本。例如,在政治说服信息的生成过程中,AI可以结合历史数据和目标受众的倾向,调整话语框架,以匹配不同意识形态群体的需求;在健康传播领域,AI则能够撰写易于理解且增强公众信任的医学建议,帮助优化健康干预策略。

在说服信息生成方面,AI(本文特指大语言模型)展现出三大核心优势 \citep{breum2024persuasive}。首先,信息组织能力使其能够整合庞大的知识库,并以清晰的逻辑结构呈现观点,从而提升信息的可读性和说服力。其次,语言风格调整能力允许AI根据不同目标受众的特征,动态调整措辞、语调和表达方式,以增强信息的吸引力和可信度。最后,高效性与一致性使AI相较于人工创作更具优势,它能够在短时间内生成大量风格统一的说服性文本,确保传播内容的稳定性。例如,在政治宣传或健康劝导信息的生成过程中,AI可以自动调整文本,使其符合目标受众的价值观和心理需求,从而提高信息的接受度。AI的说服信息生成过程主要依赖语境理解与适应性调整。首先,模型会解析输入文本的核心主题,并根据训练数据中的相似案例提取最相关的内容要素。例如,当AI接收到“撰写一篇鼓励年轻人参与环保行动的倡议书”这一指令时,它可能会识别“气候变化”“社会责任”“青年行动”等核心概念,并围绕这些主题生成劝导性文本。其次,AI会根据目标受众的特点调整语言风格和说服策略,例如,对于倾向理性分析的受众,AI可能会采用数据驱动的论证方式,而对于情感驱动型受众,则可能使用故事化叙述或道德诉求以增强共鸣。

综上所述,生成式AI凭借强大的文本生成能力,能够快速、高效地创建针对特定受众的说服性信息,并在多个领域展现出实际应用价值。这一特性不仅提升了信息传播的精准度和效率,也为个性化传播和劝导策略的优化提供了新的可能性。


\subsection{AI说服效果}
人工智能在广告说服方面的潜力得到了越来越多的实证研究支持。一项关于\textbf{AI与人类在说服力上的直接比较}的元分析研究表明(121项实验,N=53,977),总体而言AI说服效果上与人类相当。然而,AI在促进行为意图方面的说服效果略逊于人类,而在态度改变和认知层面的影响上,AI与人类的效果没有显著差异 \citep{huang2023artificial}。这一研究表明,尽管AI在广告传播中的作用日益增强,但在实际促使用户采取行动时,其效果仍然可能受到一定的限制。

随着生成式AI的发展,AI生成说服的应用更加广泛。例如,一项关于AI\textbf{生成宣传信息}的研究发现,大型语言模型(如GPT-3)能够撰写高度具有说服力的宣传文章。实验采用了8221名美国受试者,结果发现GPT-3撰写的假新闻和宣传文章在受众中具有显著的劝说效果,且经过人工筛选和优化后,其说服力在某些情况下甚至能够达到人工宣传的水平 \citep{goldstein2024persuasive}。此外,AI在\textbf{道德劝说方面}也表现出了较高的有效性。例如,一项关于AI生成道德劝说信息的研究探讨了不同道德论点在气候行动劝说中的效果。研究发现,GPT能够生成基于扩展道德基础理论的说服性气候行动声明,并且生成的内容在说服效果上普遍优于人类撰写的文本。特别是,当论点涉及同情、公平和“善待后代”等道德主题时,无论受众政治立场如何,都表现出对信息更高的接受度 \citep{nisbett2023convincing}。这一发现表明,AI能正确理解不同道德倾向的偏好,并生成更具针对性的道德劝说文本,从而提高宣传信息的影响力。在\textbf{健康传播领域},AI生成的内容同样展现出了较强的说服力。例如,一项研究评估了GPT-3生成的新冠疫苗推广信息,结果表明,AI生成的文本在信息质量、说服力和受众接受度方面均优于美国疾病控制与预防中心(CDC)发布的官方宣传材料。

尽管说服效果上差异已不明显,但在作者的感知层面 \citep{nisbett2023convincing},参与者仍评价说,相比人类作者,AI文本更加客观理性,体现为更富事实依据和逻辑性,而情绪色彩和叙事风格相对较弱。然而另一方面,人类创作在情感共鸣和创意表达上仍有优势。

\subsection{AI在个性化广告中的应用}

在广告领域中,特别是基于大型语言模型(LLMs)的个性化广告,已成为精准营销的重要手段。传统的个性化广告依赖用户的浏览历史、人口统计数据和行为特征,而LLMs的引入使得个性化广告不仅能匹配用户的兴趣,还能根据个体的心理特质进行深度定制。这一进展极大地提高了广告的说服力,使个性化广告在营销、政治传播和产品推荐等多个领域展现出前所未有的影响力。

近期研究探讨了LLMs在个性化广告中的应用及其有效性,重点关注不同人格特质(如开放性、神经质)在个性化广告中的适配性。\citet{simchon2024persuasive} 通过实验检验了AI生成的个性化政治宣传信息的说服力,研究采用相同的基础材料,并引导ChatGPT(gpt-3.5-turbo; February 2023)和GPT-3分别生成针对高开放性和低开放性个体的政治宣传信息。实验结果表明,个性化的政治宣传信息比非个性化信息更具说服力,并且随着模型的进步(GPT-3.5 相较于 GPT-3),个性化效果得到进一步增强。同样地,\citet{matz2024potential} 在多个领域(如消费品营销、政治倡导)探讨了LLMs生成个性化信息的说服力,研究方法略有不同,他们直接让AI生成个性化信息,而非基于特定材料调整生成方式。研究结果进一步证实,由GPT等模型生成的个性化信息相比非个性化信息更具说服力,表明LLMs在个性化劝导方面的潜力。此外,一些研究从文本特征的角度验证了LLMs生成个性化信息的能力。\citet{meguellati2024good} 进一步评估了LLMs在社交媒体广告和电商推荐广告中的个性化生成效果,研究重点考察了AI生成的针对高开放性(Openness)和高神经质(Neuroticism)个体的个性化广告,并与人类撰写的广告进行对比。实验结果显示,AI生成的高开放性个性化广告在用户参与度和偏好方面表现突出,其效果优于非个性化广告,甚至在某些情况下优于人类撰写的广告。然而,针对高神经质个体的个性化广告并未展现出显著优势,这表明不同人格特质对个性化广告的敏感度可能存在差异。
\citet{mieleszczenko2024dark} 探讨了基于大五人格调整语言风格以生成个性化文本的信息特征,分析了不同LLMs如何根据人格特质调整语言表达。然而,该研究主要关注文本本身的特征,而未直接测量个性化广告对受众的实际影响,缺乏用户评价数据。

总体而言,现有研究从实验和文本特征两个角度探究了LLMs在个性化广告中的应用效果。然而,不同研究在生成个性化材料的方式上有所不同,部分研究基于固定素材进行个性化改写,而另一些研究则直接让AI生成个性化信息。此外,当前的研究主要关注开放性和神经质两个特质,对其他人格维度(如宜人性、尽责性、外向性)的探索较为有限,未来研究可以进一步扩展人格特质的范围,并结合用户反馈来评估AI生成个性化广告的整体效果。

\subsection{AI作为信息源的影响}
过往有关个性化广告有效的中介机制研究中发现(详见\ref{个性化说服理论}),个性化广告之所以能够提升广告的吸引力和说服力,是因为它们有效提高了受众对广告内容的感知相似性。当受众感觉到广告内容与自身兴趣、需求、或人格特质高度契合,会缩短其与信息源之间的心理距离,进而提高信息加工的动机和广告的整体效果。然而,随着人工智能(AI)生成技术在广告领域的应用逐渐普及,一个重要的挑战也随之而来:即\textbf{算法厌恶(Algorithm Aversion)}。算法厌恶指的是当受众意识到某一建议或内容由算法生成时,往往会表现出不信任或抵触的情绪,即使该算法提供的建议质量优于人类决策 \citep{dietvorst2015algorithm}。例如,医疗研究发现,当患者得知诊断建议由AI算法提供时,即使算法的准确率高于医生,他们仍然更倾向于选择医生的建议,这正是算法厌恶的典型表现 \citep{longoni2019resistance}。算法厌恶背后的核心原因之一在于用户无法从算法身上感知到与自身的相似性,进而无法建立起有效的心理联结和情感共鸣 \citep{dang2024extended}。社会心理学中的“内群体偏好”(Ingroup Preference)提供了对此现象的深入解释:人类天然地倾向于信任与自身相似或具有相似价值观的个体(或信息源),而算法通常被视为“非人性化的外群体”,缺乏必要的情感联结和共情能力 \citep{turel2023prejudiced}。进一步从心理距离理论角度来看,人们通常更倾向于接受来自心理距离较近的信息(如朋友、家人、或与自身相似的群体成员),而算法因其“非人”特征而增加了用户与信息源之间的心理距离,进而引发用户的心理抗拒 \citep{castelo2019task}。已有实证研究发现,即使算法生成的内容在信息质量上与人类创作内容相当或更优,当受众明确得知内容由AI创作时,其对内容的可信度评价会显著降低 \citep{lin5080378visible, khan2024ai, li2024impact}。例如,在新闻传播领域,当新闻报道被明确标注为AI撰写或协作撰写时,读者普遍认为这些报道的客观性和可信度不如纯人类撰写的内容,甚至会怀疑AI信息的背后存在隐藏的操控意图。此外,当参与者不了解文本的来源时,他们更倾向于接受AI生成的信息,而当告知文本是由AI生成时,部分受众会表现出一定的抗拒 \citep{karinshak2023working}。这一研究揭示了AI在公共健康信息传播中的潜力,同时也表明,受众对AI生成内容的接受度可能受到信息来源的影响。

尽管算法厌恶现象在多个领域已被证实,但AI作为信息源在个性化广告情境下的具体影响尚未得到充分探讨,尤其是在个性化广告中明确标记内容由AI生成时,受众能否仍感知到足够的相似性并接受广告内容,目前仍是未被深入探究的研究空白。已有研究表明,当信息源、接收者及信息内容的属性高度匹配时,受众会产生更强的“正确感”或“合适感”,从而显著提高说服效果\citep{cesario2004regulatory}。具体而言,这种“匹配效应”(Matching Effect)强调了信息源与信息本身的契合对说服过程的重要性:信息源的人格、认知水平或身份特质越与受众期望的信息属性相匹配,受众对信息的认同感就越高,信息的有效性也随之增强 \citep{brinol2009source}。因此,亟需探讨当AI被明确标记为信息源时,其与个性化广告内容在抽象水平上的匹配程度如何影响用户的感知相似性和心理距离感受,并进一步验证这种匹配(或不匹配)是否会显著影响用户对个性化广告的接受度及其实际的说服效果。



\section{研究问题及框架}
本研究旨在系统地探讨人工智能(AI)生成个性化广告的有效性及其作用机制,重点关注 AI 生成广告的内容质量、个性化匹配策略,以及 AI 作为信息来源对受众心理态度的影响。现有研究表明,尽管个性化广告已广泛应用于数字营销领域,但传统的个性化广告创作主要依赖手工撰写或基于模板调整的人格化策略,难以大规模、高效地满足不同人格特质群体的需求。这一局限不仅限制了广告内容创作的灵活性,也导致在个性化匹配方面的研究深度受限。生成式人工智能的兴起,为基于人格的个性化广告创作提供了新的可能,使广告内容能够更精准地适应不同人格特质的偏好特征。然而,AI 生成的个性化广告是否真正有效、是否能够精准体现人格特质导向的个性化策略,以及 AI 作为信息来源如何影响广告接受度,仍有待系统性探讨。因此,本研究围绕 AI 生成内容与人格特质个性化广告的关系,提出了一个整合性的理论框架,并围绕三个核心问题展开研究:第一,AI 生成的个性化广告是否有效,即 AI 是否能够稳定地提升广告匹配度,并在不同人格特质群体中展现出稳定的有效性;第二,AI 在个性化广告中的执行机制,即 AI 生成的广告内容是否能够精准体现人格匹配特征,以及这些内容的语言特征如何影响广告的说服效果;第三,AI 作为信息源如何影响个性化广告的效果,即受众是否能够区分 AI 与人类专家生成的广告文本,以及 AI 作为广告创作者的身份是否会影响个性化广告的接受度。

围绕上述研究问题,本研究通过四个系统性研究,探讨 AI 生成个性化广告的适用性、执行机制及信息源影响。\textbf{研究一主要考察 AI 生成个性化广告的基本适用}性,涉及不同人格特质、产品类型及广告生成方式的影响,以验证 AI 生成广告是否能够有效匹配受众需求。本研究首先采用 GPT-3.5 生成针对不同人格特质的大五人格定制广告文本,并测量不同人格受众的说服效果。在此基础上,进一步升级至 GPT-4,并引入产品类别(实用型 vs. 享乐型)作为额外变量,探讨 AI 生成广告在不同产品类型下的适配性及效果。随后,研究进一步细化广告生成方式,比较直接基于产品描述生成与基于中性广告改编两种方法的效果,并同时考察人格特质的高低水平对 AI 生成广告匹配效果的影响。这一系列实验为 AI 在个性化广告创作中的基础能力提供了系统性的实证支持。

\textbf{研究二通过对比 AI 生成广告与人类专家创作广告,分析 AI 在个性化广告中的优势与局限性},探索 AI 是否在某些特定人格特质维度上更具表现力,以及 AI 生成内容是否存在结构性不足。首先,实验直接比较 AI(GPT-4)与人类专家撰写的个性化广告在说服效果上的差异,以明确 AI 在个性化广告创作中的整体表现。进一步地,研究引入 AI 修改人类专家文本的条件,并扩展因变量测量,包括社交媒体互动行为等更接近现实广告传播效果的指标。通过对比分析 AI 生成广告与人类专家创作广告的差异,研究进一步明确了 AI 生成广告的适用范围及其潜在局限。

\textbf{研究三采用文本分析方法,深入解析 AI 生成个性化广告的语言特征,以探讨 AI 在个性化广告中的执行机制。}本研究识别 AI 生成广告文本中的语言适配模式,重点分析 AI 如何利用不同语言特征(如认知过程词汇、情绪词汇、开放性相关表达)来匹配目标受众的人格特质。此外,研究构建预测模型,探讨哪些文本特征能够更好地预测 AI 生成广告的效果,并据此提出优化 AI 生成个性化广告的策略。该研究不仅揭示了 AI 在个性化广告创作中可能存在的偏差和局限,也为个性化广告的优化提供了数据支持。

\textbf{研究四则关注 AI 作为信息源对个性化广告的影响},探讨 AI 身份是否会影响个性化广告的接受度,并分析 AI 身份的披露如何通过感知相似性影响广告的整体说服效果。本研究首先在广告信息来源未知的情况下,测量受众对广告创作者身份的主观判断,并分析文本特征对 AI 识别度的影响。随后,研究进一步明确披露 AI 作为广告创作者,并测量这一信息披露如何影响个性化广告效果,特别是 AI 作为信息源是否会增加受众与广告之间的心理距离,进而降低个性化广告的说服力。这一研究为理解 AI 在广告传播中的双重影响提供了实证支持,即 AI 既能提升广告内容的个性化匹配度,同时其作为创作者的身份可能影响受众对广告的接受度。

综上所述,本研究通过实验与文本分析相结合的方法,系统地评估了 AI 在个性化广告创作中的有效性、语言特征与作用机制,同时深入探讨了人格特质(特别是开放性)在个性化广告匹配中的作用,并揭示了 AI 作为广告创作者的心理影响。这些研究不仅扩展了个性化广告的理论基础,也为 AI 在广告创作中的优化策略提供了重要的实践指导。



\section{研究意义}

\subsection{理论意义}

本研究通过AI得以生成大量的个性化广告文本,得以在多场景下进行系统地检验个性化广告的效果,并得以基于文本进行进一步分析。整体来说在理论层面的贡献主要体现在以下两个方面:

首先,本研究扩展了个性化广告效果的理论理解,丰富了现有基于人格特质的个性化说服模型。虽然以往的研究已提出并验证了基于人格特质的个性化广告效果,但受限于传统的人类创作方式,文本材料的数量和多样性通常不足以系统地检验个性化效果在不同场景(如不同产品类型:实用型与享乐型产品)中的表现差异。本研究利用生成式AI的优势,得以大规模、系统化地生成基于人格特质的个性化广告文本,首次在多个产品场景与多个人格维度(如开放性、神经质、外向性)下进行全面的实验检验。这不仅为现有的个性化理论模型提供了更为丰富的实证证据,也有助于更加细致地探讨不同人格之间的差异化反应。此外,本研究进一步结合了文本特征分析与受众评分,深入探讨了个性化广告文本如何在语言特征层面与受众的人格偏好形成有效匹配。这种基于大量文本材料的深入分析,有效弥补了现有文献中对个性化广告语言机制探讨不足的局限,清晰地揭示了不同人格受众在广告场景下的具体语言偏好模式,并为理解个性化广告效果差异提供了机制性解释。

此外,本研究从AI作为信息源的视角丰富了传播理论的相关研究。以往的传播学研究更多聚焦信息内容本身对受众的影响,很少系统地考察信息来源对个性化广告效果的潜在影响,尤其是在AI生成内容逐渐普及的背景下更是如此。本研究探索了消费者明确感知广告由AI生成时所产生的心理反应与态度变化,分析了AI作为广告创作者这一特殊信息源对受众的信任度、个性化感知、购买意愿所产生的影响。这一理论贡献不仅拓展了信息源在传播理论中的地位,还为未来研究AI生成内容在人际传播与媒介心理效应方面的特殊性提供了基础理论框架。

综上所述,本研究不仅深化了对个性化广告中人格差异和语言特征的理论理解,也拓展了传播理论关于AI作为信息源对受众心理影响的研究视角,为未来的研究提供了清晰的理论框架与方向。

\subsection{实践意义}

本研究在实践层面提供了多个重要启示,有助于推动个性化广告领域中AI生成技术的实际应用与优化。

首先,本研究系统性地验证了AI生成个性化广告在现实场景中的有效性,为企业和广告主提供了明确的操作路径与策略参考。研究不仅确认了AI生成的个性化广告整体上能够有效提升广告表现,还进一步检验了多种不同的生成方式,包括从零开始的直接生成、基于中性广告文本的个性化调整、基于具体产品描述的针对性创作,以及与人类专家协作生成优化版本的个性化广告。这些具体的AI内容生成策略与prompt设计,均可为企业在实际营销过程中如何利用AI进行大规模广告内容生产提供清晰的流程指引和实施方案,有助于企业以较低成本快速实现高质量、个性化内容的精准投放,提升广告的营销效率。

其次,本研究构建了基于文本特征的个性化广告效果预测模型,帮助企业和营销人员在实际广告投放之前就能准确评估广告内容的个性化表现潜力。传统的广告效果评估通常需要投入大量的资源进行A/B测试或消费者调研,费时且成本高昂。本研究开发的个性化广告预测模型利用文本语言特征快速、准确地预测特定广告在不同人格特质受众中的效果,极大提升了企业在广告前期创意阶段的决策效率和准确性。基于此模型,企业能够提前筛选、优化和确定最佳广告创意,从而减少试错成本,显著提高营销投资的回报效率。

第三,本研究揭示了AI作为广告创作者身份对消费者心理和广告效果的潜在负面影响,为行业应用AI生成内容提出了必要的预警。具体而言,当消费者意识到广告由AI生成后,其对广告的信任度和接受程度可能会受到影响。因此,企业和营销人员在实践中需要充分考虑如何巧妙管理受众对AI身份的认知,例如采取透明度适度的披露策略,或结合人工创作元素降低AI生成感,以确保受众不会因AI身份的感知而降低广告的接受意愿。此外,营销实践中也可考虑对AI内容的来源披露策略进行针对性优化,例如凸显人机协作过程或AI的辅助角色,从而缓解消费者可能产生的心理抗拒。这些具体的实践指导能够帮助营销人员更妥善地利用AI生成内容,平衡创新技术的效率与消费者心理舒适度之间的关系,推动AI在个性化广告领域的可持续应用与发展。


