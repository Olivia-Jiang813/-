\clearpage
\pagestyle{abstractstyle}     % 后续页都用摘要样式
\thispagestyle{abstractstyle} % 当前页也用摘要样式

% 如果是在 book 类的双面模式下,可能还会插入偶数空白页
% 如果不想要多余空白页,可考虑 \documentclass[oneside]{book}

\begin{center}
    {\heiti \fontsize{16pt}{16pt}\selectfont 摘\ \ \ 要}
\end{center}

% \vspace{-2cm}

在互联网与数字营销快速发展的背景下,个性化广告已成为提升广告精准度和用户体验的重要策略。传统的个性化广告通常基于用户的历史行为数据进行推荐,但这一方式受限于用户短期兴趣的波动,难以实现持久有效的个性化匹配。近年来,基于人格特质的个性化广告逐渐受到重视,因人格特质具备较高的稳定性,能够有效预测个体的偏好。然而,传统基于人格的广告设计通常面临生成成本较高、创作效率低和文案多样性不足等问题,这在一定程度上制约了个性化广告策略的推广。生成式人工智能(尤其是大语言模型)的兴起,为个性化广告的自动化创作提供了技术基础,能够高效、快速地生成匹配不同人格特质的个性化广告内容。当前对于AI生成个性化广告的研究大多停留在技术可行性验证阶段,但对于 AI 生成的个性化广告能否稳定有效地匹配受众的实际需求以及 AI 作为广告创作者的信息来源对广告接受度是否存在潜在影响,目前在学界尚未得到系统性地探讨。

针对以上问题,本研究旨在探索一条从基础可行性验证到实践情境优化的AI生成个性化广告研究路径,围绕内容生成有效性、人机协同创作模式、深层执行机制以及信息来源边界效应四个层面逐步展开。具体而言,研究一首先系统检验AI针对不同人格特质(开放性、外倾性、宜人性、尽责性和神经质)所生成的个性化广告内容的有效性,明确AI个性化内容生成的基本条件及适用范围。在确认了AI个性化广告内容的初步有效性后,研究二进一步探讨 AI 和人类专家在个性化广告创作中的差异及其各自的适配性。研究不仅直接对比 AI 和人类专家创作的个性化广告效果,还考察 AI 介入广告创作过程对最终广告表现的影响,以探索不同创作模式在个性化广告中的适用性及最优条件,从而进一步理解人机协同在个性化广告创作中的潜力。随后,研究三深入分析AI生成个性化广告的执行机制,采用文本分析方法识别AI生成文本中的语言特征,例如认知过程词汇、情绪词汇、开放性相关表达等,以系统评估AI如何在个性化广告创作中执行人格特质匹配。此外,研究构建预测模型,以识别哪些文本特征能够更好地预测个性化广告的效果,并据此提出优化AI生成广告的策略。在此基础上,研究四聚焦于信息源身份对广告接受度的影响,探索受众对广告创作者身份(AI与人类)的认知如何调节广告效果。具体而言,研究通过受众在信息来源未知与明确AI创作者身份两种条件下的对比,考察AI身份认知如何影响个性化广告的感知相似性与说服效果,从而揭示AI作为创作者在个性化广告投放情境中的边界作用。

通过上述系列研究本篇论文得出以下主要结论:

1. AI在开放性与外倾性维度生成的个性化广告效果稳定,而在宜人性和尽责性维度则受人格水平与广告生成方式影响明显。神经质维度的个性化效果整体不显著。

2. AI与人类专家在个性化广告创作中优势互补。AI更适合于生成逻辑严密、结构清晰的信息;人类则更善于情感共鸣型的广告创作,两者的结合能够提升个性化广告效果。

3. 个性化广告的有效性不仅依赖于语言风格的匹配,更取决于是否准确地挖掘广告语境中受众的深层需求。AI个性化广告优化需要考虑受众在具体情境下的认知加工模式,而不仅停留于表层特征的复制。

4. 受众对AI作为广告创作者的认知影响个性化广告的接受度。受众在未知信息来源的情况下能够在一定程度上区分AI生成广告,但判断并不总是准确。当AI身份被明确披露后,广告的说服力下降,受众的感知相似性降低,从而削弱广告的整体说服效果。

本研究的理论贡献在于提出并系统验证了AI与个性化广告的整合路径,涵盖基础技术验证、人机协同、深层需求匹配以及信息来源效应,全面丰富并深化了个性化广告的理论框架。在实践层面,本研究提出了以心理学为基础的AI个性化广告优化策略:(1)优化AI提示词设计,兼顾语言风格与受众认知需求;(2)推行人机协同创作模式,综合发挥AI和人类各自优势;(3)策略性调整AI身份披露方式,以降低信息来源效应对广告接受度的负面影响。这些策略为AI技术在广告领域的实际应用提供了明确而可行的实践指导。

\vspace{1cm}

\textbf{关键词:} 个性化广告,大五人格,AI,感知相似性,说服

% 摘要结束后 强制切换成空白样式,防止影响下一页(目录)
\clearpage
\pagestyle{empty}
\thispagestyle{empty}
