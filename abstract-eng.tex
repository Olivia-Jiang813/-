\clearpage
\pagestyle{abstractstyle-eng}     % 后续页都用摘要样式
\thispagestyle{abstractstyle-eng} % 当前页也用摘要样式

% 如果是在 book 类的双面模式下,可能还会插入偶数空白页
% 如果不想要多余空白页,可考虑 \documentclass[oneside]{book}

\begin{center}
    \textbf{{\fontsize{16pt}{16pt}\selectfont Abstract}}
\end{center}

% \vspace{-2cm}

In the context of rapid advancements in the internet and digital marketing, personalized advertising has become a key strategy for improving ad accuracy and enhancing user experience. Traditional personalized advertising primarily relies on users’ historical behavioral data for recommendations. However, this approach is constrained by short-term fluctuations in user interests, making it difficult to achieve long-lasting personalized matching. In recent years, personality-based personalized advertising has gained increasing attention, as personality traits exhibit high stability and can effectively predict individual preferences. However, traditional personality-based advertising design often faces challenges such as high content generation costs, low creative efficiency, and limited diversity in ad copy, which hinder the widespread adoption of personalized advertising strategies. The emergence of generative artificial intelligence, particularly large language models, has provided a technological foundation for automating personalized ad creation, enabling the rapid and efficient generation of ad content tailored to different personality traits. However, existing research on AI-generated personalized advertising remains largely focused on technological feasibility, lacking a systematic examination of key issues ranging from content creation models to audience acceptance mechanisms. In particular, significant research gaps remain regarding the effectiveness of AI-generated personalized advertising, the division of roles between AI and human creators, the alignment between AI-generated content and audience needs, and the effects of AI as an information source.  

To address these gaps, this study explores the pathways for integrating AI into personalized advertising, systematically investigating the process from AI-generated content validation to the role positioning of AI and human experts, audience preference alignment, and the psychological mechanisms underlying AI as an information source. The aim is to establish a comprehensive research framework that spans from content creation to audience acceptance. Specifically, Study 1 systematically examines the effectiveness of AI-generated personalized ads across different personality traits, including openness, extraversion, conscientiousness, agreeableness, and neuroticism, considering different personality levels and comparing two advertising creation methods: adaptation from neutral advertisements and direct generation from product descriptions. This foundational validation clarifies the effectiveness of AI-generated content and its conditions. Study 2 further explores the differences in personalized ad effectiveness between AI and human experts through a comparative experiment, aiming to identify the optimal division of labor between AI and human creators in human-AI collaborative advertising. Study 3 delves deeper into the alignment between AI-generated content and audiences' cognitive needs, uncovering potential limitations of AI in capturing deep audience preferences and advancing the exploration toward content-depth matching. Study 4 focuses on the effects of AI as an information source, investigating how audience awareness of AI as the ad creator influences psychological responses and ad acceptance, thus completing the research framework from content creation to audience reception.  

Based on this series of studies, several key conclusions are drawn. First, AI-generated personalized ads exhibit stable effectiveness for openness and extraversion, whereas their effectiveness in agreeableness and conscientiousness is significantly influenced by personality levels and ad generation methods. The personalization effect for neuroticism is generally non-significant. Second, AI and human experts demonstrate complementary strengths in personalized ad creation. AI excels in generating logically structured and information-dense content, while human experts are more adept at crafting emotionally resonant advertisements. The integration of both approaches enhances personalized ad effectiveness. Third, the effectiveness of personalized advertising is not solely dependent on linguistic style matching but also on the accurate identification of deep audience needs within the advertising context. Optimizing AI-generated personalized advertising requires an understanding of audience cognitive processing patterns in specific contexts rather than merely replicating surface-level features. Fourth, audience perceptions of AI as an ad creator significantly influence the acceptance of personalized advertising. While audiences can, to some extent, distinguish AI-generated ads from human-created ones when the source is undisclosed, their judgments are not always accurate. When AI identity is explicitly disclosed, ad persuasiveness declines, perceived similarity decreases, and overall advertising effectiveness is weakened.  

This study makes several theoretical contributions by systematically proposing and validating the integration pathways between AI and personalized advertising. From feasibility validation, role positioning, and human-AI collaboration to deep audience preference matching and information source effects, this research expands and deepens the theoretical framework of personalized advertising. In practical terms, this study proposes AI-driven personalized advertising optimization strategies based on psychological principles. First, optimizing AI prompt design should balance linguistic style matching with audience cognitive needs. Second, promoting a human-AI collaborative creation model can leverage the complementary advantages of AI and human creators. Third, strategically adjusting AI identity disclosure can mitigate the negative effects of information source perception on ad acceptance. These strategies provide clear and actionable guidance for the practical application of AI in the advertising industry.

\vspace{1cm}

\textbf{Keyword:} Personalized Advertising,Big Five Personality,AI,Perceived Similarity,Persuasion

% 摘要结束后 强制切换成空白样式,防止影响下一页(目录)
\clearpage
\pagestyle{empty}
\thispagestyle{empty}
