\section{研究问题及框架}
本研究旨在系统地探讨人工智能(AI)生成个性化广告的有效性及其作用机制,重点关注 AI 生成广告的内容质量、个性化匹配策略,以及 AI 作为信息来源对受众心理态度的影响。现有研究表明,尽管个性化广告已广泛应用于数字营销领域,但传统的个性化广告创作主要依赖手工撰写或基于模板调整的人格化策略,难以大规模、高效地满足不同人格特质群体的需求。这一局限不仅限制了广告内容创作的灵活性,也导致在个性化匹配方面的研究深度受限。生成式人工智能的兴起,为基于人格的个性化广告创作提供了新的可能,使广告内容能够更精准地适应不同人格特质的偏好特征。然而,AI 生成的个性化广告是否真正有效、是否能够精准体现人格特质导向的个性化策略,以及 AI 作为信息来源如何影响广告接受度,仍有待系统性探讨。因此,本研究围绕 AI 生成内容与人格特质个性化广告的关系,提出了一个整合性的理论框架,并围绕三个核心问题展开研究:第一,AI 生成的个性化广告是否有效,即 AI 是否能够稳定地提升广告匹配度,并在不同人格特质群体中展现出稳定的有效性;第二,AI 在个性化广告中的执行机制,即 AI 生成的广告内容是否能够精准体现人格匹配特征,以及这些内容的语言特征如何影响广告的说服效果;第三,AI 作为信息源如何影响个性化广告的效果,即受众是否能够区分 AI 与人类专家生成的广告文本,以及 AI 作为广告创作者的身份是否会影响个性化广告的接受度。

围绕上述研究问题,本研究通过四个系统性研究,探讨 AI 生成个性化广告的适用性、执行机制及信息源影响。\textbf{研究一主要考察 AI 生成个性化广告的基本适用}性,涉及不同人格特质、产品类型及广告生成方式的影响,以验证 AI 生成广告是否能够有效匹配受众需求。本研究首先采用 GPT-3.5 生成针对不同人格特质的大五人格定制广告文本,并测量不同人格受众的说服效果。在此基础上,进一步升级至 GPT-4,并引入产品类别(实用型 vs. 享乐型)作为额外变量,探讨 AI 生成广告在不同产品类型下的适配性及效果。随后,研究进一步细化广告生成方式,比较直接基于产品描述生成与基于中性广告改编两种方法的效果,并同时考察人格特质的高低水平对 AI 生成广告匹配效果的影响。这一系列实验为 AI 在个性化广告创作中的基础能力提供了系统性的实证支持。

\textbf{研究二通过对比 AI 生成广告与人类专家创作广告,分析 AI 在个性化广告中的优势与局限性},探索 AI 是否在某些特定人格特质维度上更具表现力,以及 AI 生成内容是否存在结构性不足。首先,实验直接比较 AI(GPT-4)与人类专家撰写的个性化广告在说服效果上的差异,以明确 AI 在个性化广告创作中的整体表现。进一步地,研究引入 AI 修改人类专家文本的条件,并扩展因变量测量,包括社交媒体互动行为等更接近现实广告传播效果的指标。通过对比分析 AI 生成广告与人类专家创作广告的差异,研究进一步明确了 AI 生成广告的适用范围及其潜在局限。

\textbf{研究三采用文本分析方法,深入解析 AI 生成个性化广告的语言特征,以探讨 AI 在个性化广告中的执行机制。}本研究识别 AI 生成广告文本中的语言适配模式,重点分析 AI 如何利用不同语言特征(如认知过程词汇、情绪词汇、开放性相关表达)来匹配目标受众的人格特质。此外,研究构建预测模型,探讨哪些文本特征能够更好地预测 AI 生成广告的效果,并据此提出优化 AI 生成个性化广告的策略。该研究不仅揭示了 AI 在个性化广告创作中可能存在的偏差和局限,也为个性化广告的优化提供了数据支持。

\textbf{研究四则关注 AI 作为信息源对个性化广告的影响},探讨 AI 身份是否会影响个性化广告的接受度,并分析 AI 身份的披露如何通过感知相似性影响广告的整体说服效果。本研究首先在广告信息来源未知的情况下,测量受众对广告创作者身份的主观判断,并分析文本特征对 AI 识别度的影响。随后,研究进一步明确披露 AI 作为广告创作者,并测量这一信息披露如何影响个性化广告效果,特别是 AI 作为信息源是否会增加受众与广告之间的心理距离,进而降低个性化广告的说服力。这一研究为理解 AI 在广告传播中的双重影响提供了实证支持,即 AI 既能提升广告内容的个性化匹配度,同时其作为创作者的身份可能影响受众对广告的接受度。

综上所述,本研究通过实验与文本分析相结合的方法,系统地评估了 AI 在个性化广告创作中的有效性、语言特征与作用机制,同时深入探讨了人格特质(特别是开放性)在个性化广告匹配中的作用,并揭示了 AI 作为广告创作者的心理影响。这些研究不仅扩展了个性化广告的理论基础,也为 AI 在广告创作中的优化策略提供了重要的实践指导。



\section{研究意义}

\subsection{理论意义}

本研究通过AI得以生成大量的个性化广告文本,得以在多场景下进行系统地检验个性化广告的效果,并得以基于文本进行进一步分析。整体来说在理论层面的贡献主要体现在以下两个方面:

首先,本研究扩展了个性化广告效果的理论理解,丰富了现有基于人格特质的个性化说服模型。虽然以往的研究已提出并验证了基于人格特质的个性化广告效果,但受限于传统的人类创作方式,文本材料的数量和多样性通常不足以系统地检验个性化效果在不同场景(如不同产品类型:实用型与享乐型产品)中的表现差异。本研究利用生成式AI的优势,得以大规模、系统化地生成基于人格特质的个性化广告文本,首次在多个产品场景与多个人格维度(如开放性、神经质、外向性)下进行全面的实验检验。这不仅为现有的个性化理论模型提供了更为丰富的实证证据,也有助于更加细致地探讨不同人格之间的差异化反应。此外,本研究进一步结合了文本特征分析与受众评分,深入探讨了个性化广告文本如何在语言特征层面与受众的人格偏好形成有效匹配。这种基于大量文本材料的深入分析,有效弥补了现有文献中对个性化广告语言机制探讨不足的局限,清晰地揭示了不同人格受众在广告场景下的具体语言偏好模式,并为理解个性化广告效果差异提供了机制性解释。

此外,本研究从AI作为信息源的视角丰富了传播理论的相关研究。以往的传播学研究更多聚焦信息内容本身对受众的影响,很少系统地考察信息来源对个性化广告效果的潜在影响,尤其是在AI生成内容逐渐普及的背景下更是如此。本研究探索了消费者明确感知广告由AI生成时所产生的心理反应与态度变化,分析了AI作为广告创作者这一特殊信息源对受众的信任度、个性化感知、购买意愿所产生的影响。这一理论贡献不仅拓展了信息源在传播理论中的地位,还为未来研究AI生成内容在人际传播与媒介心理效应方面的特殊性提供了基础理论框架。

综上所述,本研究不仅深化了对个性化广告中人格差异和语言特征的理论理解,也拓展了传播理论关于AI作为信息源对受众心理影响的研究视角,为未来的研究提供了清晰的理论框架与方向。

\subsection{实践意义}

本研究在实践层面提供了多个重要启示,有助于推动个性化广告领域中AI生成技术的实际应用与优化。

首先,本研究系统性地验证了AI生成个性化广告在现实场景中的有效性,为企业和广告主提供了明确的操作路径与策略参考。研究不仅确认了AI生成的个性化广告整体上能够有效提升广告表现,还进一步检验了多种不同的生成方式,包括从零开始的直接生成、基于中性广告文本的个性化调整、基于具体产品描述的针对性创作,以及与人类专家协作生成优化版本的个性化广告。这些具体的AI内容生成策略与prompt设计,均可为企业在实际营销过程中如何利用AI进行大规模广告内容生产提供清晰的流程指引和实施方案,有助于企业以较低成本快速实现高质量、个性化内容的精准投放,提升广告的营销效率。

其次,本研究构建了基于文本特征的个性化广告效果预测模型,帮助企业和营销人员在实际广告投放之前就能准确评估广告内容的个性化表现潜力。传统的广告效果评估通常需要投入大量的资源进行A/B测试或消费者调研,费时且成本高昂。本研究开发的个性化广告预测模型利用文本语言特征快速、准确地预测特定广告在不同人格特质受众中的效果,极大提升了企业在广告前期创意阶段的决策效率和准确性。基于此模型,企业能够提前筛选、优化和确定最佳广告创意,从而减少试错成本,显著提高营销投资的回报效率。

第三,本研究揭示了AI作为广告创作者身份对消费者心理和广告效果的潜在负面影响,为行业应用AI生成内容提出了必要的预警。具体而言,当消费者意识到广告由AI生成后,其对广告的信任度和接受程度可能会受到影响。因此,企业和营销人员在实践中需要充分考虑如何巧妙管理受众对AI身份的认知,例如采取透明度适度的披露策略,或结合人工创作元素降低AI生成感,以确保受众不会因AI身份的感知而降低广告的接受意愿。此外,营销实践中也可考虑对AI内容的来源披露策略进行针对性优化,例如凸显人机协作过程或AI的辅助角色,从而缓解消费者可能产生的心理抗拒。这些具体的实践指导能够帮助营销人员更妥善地利用AI生成内容,平衡创新技术的效率与消费者心理舒适度之间的关系,推动AI在个性化广告领域的可持续应用与发展。
