\section{讨论}
本研究通过四个实验系统性地探讨了AI基于大五人格特质生成个性化广告的能力,重点关注其在不同人格特质群体中的有效性,并考察了AI在不同广告创作情境下的适用性。研究结果提供了对 AI 生成广告个性化能力的初步验证。

首先\textbf{实验1}作为AI生成个性化广告的初步验证,采用GPT-3.5生成针对大五人格五个维度的个性化广告,结果显示针对\textit{高外倾性}和\textit{高宜人性}设计的个性化广告具有显著的说服效果,\textit{高开放性}呈边缘显著;而\textit{高尽责性}、\textit{高神经质}的广告效果并不显著。这表明AI在某些人格特质群体(如高宜人性和高外倾性)中能够有效生成个性化广告,而在其他人格特质(如高尽责性和高神经质)中,其个性化效果仍存在一定局限。在\textbf{实验2}中,我们进一步考察了AI在不同产品场景下的个性化广告生成能力,并采用GPT-4进行广告创作。实验结果表明,相较于实验1中的 GPT-3.5,GPT-4 能够更有效地生成针对高开放性和高尽责性个体的个性化广告,并在不同产品类型(享乐型 vs. 实用型)中展现出稳定的个性化广告生成能力。这一结果表明,AI 不仅可以基于大五人格进行个性化创作,还能够根据产品类型调整个性化策略,进一步增强广告的针对性和说服力。此外,该实验还验证了AI在基于中性广告进行个性化改写的能力,说明AI可以在已有广告文本的基础上进行个性化调整,而不仅仅是从零生成个性化广告。相较于实验1采用的 GPT-3.5,GPT-4在生成个性化广告方面表现出更强的能力,尤其在开放性和尽责性个体中,其个性化广告的说服力得到了显著提升。这一结果不仅说明 AI 生成的个性化广告效果在不同产品类别中具有一定的稳定性,也表明随着 AI 语言模型能力的提升,其个性化生成效果可能进一步优化。

虽然已有研究表明个性化广告的有效性,但大多数研究仍局限于针对各人格特质的高水平设计 \citep[如][]{hirsh2012personalized,matz2017psychological,winter2021effects}),而对低水平人格特质的个性化广告研究仍然有限,且缺乏系统性探讨。本研究的\textbf{实验3和实验4}进一步考察了AI在不同人格特质水平(高 vs. 低)下的个性化广告生成能力。\textbf{实验3} 结果表明,在\textit{开放性、外倾性和宜人性}维度上,AI生成的个性化广告能够有效匹配目标受众的个性特质,即高水平个体更偏好针对高水平特质设计的广告,低水平个体更偏好针对低水平特质设计的广告。然而,在尽责性维度上,并未观察到显著的匹配效应,这表明尽责性个体的广告偏好可能受其他因素影响,或现有个性化策略仍需进一步优化。此外,关键词分析结果进一步验证了个性化广告的有效性,发现高水平特质个体在广告中更关注符合自身人格特质的关键描述,而低水平特质个体则更倾向于与其匹配的词语特征。在\textbf{实验4}中,我们探讨了AI在基于产品描述直接生成个性化广告的能力,考察其在缺乏中性广告的情况下,是否能够生成符合不同人格特质的个性化广告。结果显示,在\textit{开放性和外倾性}维度上,AI生成的个性化广告仍然表现出显著的匹配效应,高水平特质个体更偏好高水平个性化广告,低水平特质个体则更偏好低水平个性化广告。然而,在\textit{尽责性}维度,个性化效果呈现负向匹配效应,即高尽责性个体更偏好针对低尽责性设计的广告,而低尽责性个体更偏好针对高尽责性设计的广告,这可能与尽责性个体的审慎决策模式有关。此外,在\textit{宜人性维度}上,并未观察到显著的个性化效果,可能是由于宜人性个体的广告偏好差异较小,或是由于实验样本在高宜人性个体中存在较大异质性,影响了个性化匹配的稳定性。这一结果提示,未来研究在优化个性化广告的生成策略时,应进一步考察不同人格特质群体在广告偏好上的具体特征,并结合精细化的提示语(prompt)优化AI生成的广告内容,以提高个性化效果的精准度。

综合来看,本研究对个性化广告的研究进行了补充和拓展。首先,\textbf{实验1和实验2} 说明了AI生成的个性化广告在\textbf{部分人格特质群体}中的有效性,这与\citet{hirsh2012personalized} 和\citet{matz2017psychological} 的结论一致,同时也为\citet{winter2021effects} 提出的个性化广告可能不总是有效的观点提供了新的解释。其次,通过\textbf{实验3和实验4} 的探讨,本研究进一步揭示了\textbf{个性化广告在不同人格水平群体中的匹配效应},并发现尽责性维度的负向匹配效应可能源于广告内容设计与目标群体需求的不匹配。此外,在已有的AI生成个性化广告研究(如\citet{matz2024potential})的基础上,本研究更系统地考察了\textbf{AI在多个生成场景(如基于中性广告改写、基于产品描述直接生成)的个性化能力},并引入了高低人格水平的实验设计,以更全面地检验AI生成个性化广告的有效性。

此外,无论是本研究的研究一,还是既有关于AI在说服性文本生成方面的研究 \citep[如][]{bai2023artificial,goldstein2024persuasive},均主要聚焦于AI本身的能力,而较少涉及AI在与人类专家的直接比较中的表现。然而,以往的说服性研究多基于人类专家或个体创作的文本,因此,将AI与人类专家进行对比不仅能够提供更精确的基准,以衡量AI在个性化广告创作中的实际水平,还能进一步揭示其在不同情境下的适用性与局限性。人类专家长期以来被视为广告创作的标准,其具备更丰富的市场洞察力、更复杂的创意表达方式以及更强的受众情感共鸣能力,而AI则凭借高效的文本生成能力和个性化调整优势,在广告创作中展现出潜在价值。因此,明确AI生成的个性化广告在说服效果上是否能够达到人类专家的水平,或者在人格化定制的特定场景下是否能够展现出特定优势,是一个值得深入探讨的问题。

此外,研究一的结果表明,AI在不同人格特质条件下的个性化广告生成能力并不均衡。例如,在尽责性和宜人性维度上的个性化效果未能稳定显现,甚至在尽责性维度上出现了负向匹配效应。这一现象进一步引发了对于个性化广告稳定性的讨论,即人类专家创作的广告是否能在不同人格特质条件下展现出更稳定的说服效果? 现有文献亦表明,AI的说服效能可能因任务性质和应用情境的不同而存在差异。例如,\citet{huang2023artificial} 研究发现,AI在塑造行为意图方面的效果低于人类专家,但在影响个体感知、态度和实际行为方面,与人类专家并无显著差异。因此,在个性化广告这一特定应用场景下,AI与人类专家在说服力上的比较仍然缺乏系统性的实证研究。AI是否能在广告个性化定制中与专家的创作水平相匹配?不同人格特质群体对于AI与专家创作的广告是否存在差异化的接受度?这些问题将在研究二中展开系统性探讨,以进一步评估AI在个性化广告中的优势与局限,并深化对AI生成广告在实际市场应用中的有效性及边界的理解。