\section{实验1:AI生成个性化广告的初步检验}

\subsection{方法}
本实验旨在检验针对大五人格特质进行个性化广告设计的效果。在引导AI生成广告时,选用实验进行时表现最优的模型GPT-3.5。广告内容依据五种不同人格维度(外倾性、开放性、尽责性、宜人性、神经质)的高水平特质进行个性化设计,即分别为高外倾、高开放、高尽责、高宜人和高神经质的消费者进行个性化设计。GPT-3.5针对每种人格水平生成了3则广告,每位参与者在对应人格水平条件下随机观看其中一则广告。参与者依次观看针对大五人格五个维度设计的个性化广告,广告呈现顺序随机,且需对每则广告进行评价。

\textbf{(1)被试}

通过见数平台发布实验,200名参与者自愿参加这项研究。8名参与者由于注意检查测试未通过被剔除,剩余\textbf{192名}有效参与者(年龄范围= 18-58岁;\textit{M}=29.50岁;\textit{SD}=7.83;女性109名)。每名参与者完成实验后获得1元人民币作为报酬。注意力检测包含两部分,分别嵌入在因变量测量和人格问卷中,以明确指令题形式呈现(如“请选2”)。参与者需在两道注意力检测题中均作答正确,方可被纳入有效数据样本。

\textbf{(2)实验材料:广告设计}

实验共设计\textbf{5(广告人格:外倾性/开放性/尽责性/宜人性/神经质}的高水平设计共5个条件生成广告材料,采用GPT3.5生成。根据前人文献选择中性的产品手机,广告描述避免具体品牌名以排除品牌的影响。针对每个条件,GPT-3.5生成了5则广告文案,初始生成了25则广告材料。

在生成广告时,向GPT提供了结构化的提示语(prompt),包括目标消费者特点的描述、广告设计要求和基本特征描述。例如针对高外倾条件时的prompt如表\ref{tab:study1-exp-prompt}所示。提示语要求广告文案符合特定特质群体的需求,同时强调广告的说服力和针对性的语言表达。

\begin{table}[H]
    \caption{\label{tab:study1-exp-prompt} Prompt示例}
    {\tablesongti % 整个表格环境应用宋体六号字体
    \renewcommand{\arraystretch}{1} % 调整行距
    \begin{tabularx}{\linewidth}{lXc>{\centering\arraybackslash}X}
        \toprule
        \textbf{维度} & \textbf{Prompt} \\
        \midrule
        外倾性 & 请写五段有说服力的手机广告文案(每一段30字左右),需要满足下列要求: \newline
        1)广告呈现在社交媒体上(如小红书,微博等);\newline
        2)目标消费者特点是健谈的、精力充沛的、善于交际的、外向的、喜欢与他人互动的;\newline
        3)广告内容需要结合目标消费者特点和手机产品的特点;\newline
        4)广告目的是增加消费者购买的意愿,提高广告转化率。\newline
        \textbf{其他注意事项}:\newline
        避免出现“健谈”、“精力充沛”、“交际”、“外向”、“互动”等词语。\newline
        每一段可以变换一下句式。\newline
        不出现手机品牌。\newline
        注意是针对手机的广告(新增,担心文字多了GPT忘记是针对手机的广告)。 \\
        \bottomrule
    \end{tabularx}}
\end{table}

为验证生成的个性化广告是否能够有效传达针对目标消费者的特质,本研究在正式实验前通过见数平台发布了预实验,共有110名参与者自愿参加,完成实验后获得0.5元人民币作为报酬。预实验的筛选基于以下两个标准:1)目标消费者特质匹配性。参与者需对25则广告的目标消费者进行选择,任务为从5(人格特质)的高水平描述中选择最匹配的目标消费者。对于每则广告,若设计时对应的人格特质在选择中占比最高,则该广告被视为有效。例如,若广告是针对高外倾性设计的,而在选择结果中“高外倾性”选项占比最多,则表明该广告成功传达了高外倾性的特质;2)广告本身有效性。要求参与者对广告的理解程度与相关程度进行评分,采用1-5点量表,其中评分大于3分的广告被视为有效。经过预实验筛选,剩余15条(详见附录)。


\textbf{(3)问卷测量}
\label{study1-substudy1-measurement}

a. 大五人格量表。采用\citet{rammstedt2007measuring}编制的简版大五人格量表(Big Five Inventory,BFI-10,如表\ref{tab:BFI10-sample})测量参与者的人格特征。该量表共 10个项目,分属外倾性、宜人性、神经质、开放性、尽责性5个维度,每个维度有2道题目测量。参与者在1-5点李克特式量表上进行评分(选项“1”代表“非常不符合”,“5”代表“非常符合”)。

\begin{table}[H]
    \caption{\label{tab:BFI10-sample} BFI-10量表题目及计分方式}
    {\tablesongti % 整个表格环境应用宋体六号字体
    \renewcommand{\arraystretch}{1} % 调整行距
    \begin{tabularx}{\linewidth}{lXc>{\centering\arraybackslash}X}
        \toprule % 表格顶线
        序号 & 题目 & 人格 & 计分方式 \\ 
        \midrule % 表头下的中线
        1 & 总体而言是信任他人的 & 宜人性 & 正向 \\
        2 & 喜欢寻找别人的缺点   & 宜人性 & 反向 \\
        3 & 话不多              & 外倾性 & 反向 \\
        4 & 开朗,社交能力强     & 外倾性 & 正向 \\
        5 & 容易紧张或焦虑       & 神经质 & 正向 \\
        6 & 抗压能力强,容易放松 & 神经质 & 反向 \\
        7 & 想象力丰富           & 开放性 & 正向 \\
        8 & 对艺术不怎么感兴趣   & 开放性 & 反向 \\
        9 & 工作细致周到         & 尽责性 & 正向 \\
        10 & 懒惰               & 尽责性 & 反向 \\
        \bottomrule % 表格底线
    \end{tabularx}}
\end{table}

b. 说服效果。采用\citet{hirsh2012personalized}使用的五道题目测量广告的说服效果(表\ref{tab:persuasionSurvey}),包含参与者对广告的态度,以及对广告中产品的购买意愿。参与者需要在 5 点李克特式量表上对有关表述的同意程度进行评分(选项“1”代表“非常不同意”,“5”代表“非常统一”)。

\begin{table}[htbp]
    \caption{\label{tab:persuasionSurvey}广告说服效果量表}
    {\tablesongti % 整个表格环境应用宋体六号字体
    \renewcommand{\arraystretch}{1.5} % 调整行距
    \begin{tabularx}{\linewidth}{lXc>{\centering\arraybackslash}X}
        \toprule % 表头上方粗线
        序号 & 类别 & 题目 \\ 
        \midrule % 表头下方中线
        1 & 广告态度 & 这则广告让我对这个商品更感兴趣 \\ 
        2 & 广告态度 & 这则广告让我更想了解这个商品 \\ 
        3 & 广告态度 & 总的来说,我喜欢这则广告 \\ 
        4 & 购买意愿 & 我会考虑购买这个商品 \\ 
        5 & 购买意愿 & 如果我有需要,我会购买这个商品 \\ 
        \bottomrule % 表底粗线
    \end{tabularx}
    }
\end{table}

\subsection{实验流程}
本实验的流程分为两个部分。第一部分,参与者依次阅读每则广告,共5则广告,并对每组广告的相对说服效果进行评分。广告呈现顺序随机化,以控制顺序效应。第二部分,参与者需回答与人格测试相关的问卷题目,最后提供年龄、性别等人口统计学信息。

\subsection{结果}
为检验 AI(GPT-3.5)生成的个性化广告的有效性,本实验分别对每个特质的结果进行了回归分析。回归模型的自变量为参与者对应的人格特质得分(尽责性、开放性、外倾性、宜人性、神经质),因变量为个性化广告的说服效果。结果如表 \ref{tab:study1_traitResults} 所示。

回归分析结果表明,针对\textbf{宜人性} ($\beta = 0.2624, \textit{p} = 0.004$) 和 \textbf{外倾性} ($\beta = 0.1987, \textit{p} = 0.003$) 设计的个性化广告具有显著的说服效果,即个体的宜人性和外倾性水平越高,个性化广告的说服效果越好。此外,\textbf{开放性} 个体对个性化广告的接受度也呈现边缘显著的正向关系 ($\beta = 0.1394, \textit{p} = 0.060$),表明高开放性个体可能对AI生成的个性化广告表现出一定的偏好,但该效应仍需进一步检验。相比之下,\textbf{神经质} ($\beta = -0.1422, \textit{p} = 0.172$) 和 \textbf{尽责性} ($\beta = 0.0875, \textit{p} = 0.244$) 维度的个性化广告并未表现出显著的说服效果。整体而言,本研究的回归分析结果支持了 AI 生成的个性化广告在特定人格特质群体(如高宜人性和高外倾性个体)中的有效性,但其在特定人群中(如神经质,尽责性、开放性)的适用性和优化策略仍需进一步探讨。

\begin{table}[H]
    \centering
    \begin{threeparttable}
        \caption{\label{tab:study1_traitResults} 人格特质回归分析结果}
        {\tablesongti
        \renewcommand{\arraystretch}{1}
        \begin{tabular}{p{2cm} c c c c c c} 
            \toprule
            人格特质 & 系数 & 标准误差 & \textit{t} & \textit{P} $>|t|$ & [0.025 & 0.975] \\ 
            \midrule
            \textbf{宜人性} & \textbf{0.2624} & \textbf{0.091} & \textbf{2.876} & \textbf{0.004}\textsuperscript{**} & \textbf{0.082} & \textbf{0.442} \\
            \textbf{外倾性} & \textbf{0.1987} & \textbf{0.067} & \textbf{2.970} & \textbf{0.003}\textsuperscript{**} & \textbf{0.066} & \textbf{0.331} \\
            神经质 & -0.1422 & 0.104 & -1.370 & 0.172 & -0.348 & 0.063 \\
            尽责性 & 0.0875 & 0.075 & 1.167 & 0.244 & -0.061 & 0.236 \\
            \textbf{开放性} & \textbf{0.1394} & \textbf{0.074} & \textbf{1.894} & \textbf{0.060}\textsuperscript{\dag} & \textbf{-0.006} & \textbf{0.285} \\
            \bottomrule
        \end{tabular}%
        }% 结束 \tablesongti 的作用范围
        % 让注解左对齐(flushleft),表格整体仍保持居中
        \begin{tablenotes}[flushleft]
            \footnotesize
            \item 注:*** $p < 0.001$,** $p < 0.01$,* $p < 0.05$,\textsuperscript{\dag} $p < 0.1$。
        \end{tablenotes}
    \end{threeparttable}
\end{table}









