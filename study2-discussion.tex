\section{讨论}

本研究通过实验1和实验2系统地比较了AI(GPT-4)与人类专家在个性化广告创作中的表现,并进一步探讨了AI优化人类专家广告文案的潜力。结果表明,AI能够在多个人格特质维度上生成有效的个性化广告,且在某些特质(如尽责性)上,其个性化效果甚至优于人类专家。此外,当AI用于优化人类专家文案时,个性化广告的效果在宜人性维度得到了显著提升。这一发现不仅为AI在个性化广告领域的应用提供了新的证据,也为更广泛的说服与信息传播研究提供了重要启示 \citep{dehnert2022persuasion}。

尽责性和宜人性两个特质的个性化广告在实验结果中展现出了一定的复杂性。在尽责性维度上,GPT-4 生成的广告表现出稳定的个性化匹配效果,而人类专家的个性化效果则不显著,且GPT-4和人类专家的效果在实验1中存在边缘显著的差异。这一结果表明,AI在尽责性个性化广告的创作上具有更强的适配性,而人类专家在这一维度的广告效果则存在较大的不确定性。尽责性人格的核心特征包括自律、责任感和条理性\citep{roberts2014conscientiousness},这类个体通常偏好结构清晰、信息明确、逻辑严谨的广告内容。然而,人类专家在撰写广告时可能受到自身写作风格的影响,使得广告文本在逻辑性和条理性上存在个体差异,导致尽责性个性化广告的效果不稳定。而GPT-4 作为基于大规模语料训练的AI模型,能够在生成过程中更一致地遵循逻辑清晰、信息直接的模式,因此其尽责性个性化广告更具稳定性。这也说明,通过优化prompt以及随着模型能力的提升,AI在尽责性个性化广告创作中的适应性可能仍在增强。

在宜人性维度上,实验2的结果显示,GPT-4 修改人类专家文案后,其个性化广告在说服效果和微博互动意愿方面均显著优于人类专家组,而人类专家组本身的个性化效果未达到显著水平。这一发现表明,在宜人性个性化广告的创作过程中,AI不仅能够独立生成有效的个性化广告,还能够优化人类专家的广告,使其更符合目标受众的需求。宜人性人格的个体通常更加注重社交互动、合作与温暖的表达\citep{graziano2007agreeableness},这意味着个性化广告需要更强的情感共鸣和人际取向。然而,人类专家可能并未能充分把握宜人性个体的偏好,导致其广告在实验间表现出较大的不稳定性。而AI在优化人类专家广告时,能够更系统性地强化广告中的情感元素和亲和力表达,使其更符合高宜人性个体的预期。这表明,在个性化广告创作中,AI不仅能够弥补人类专家可能存在的个性化表达不足,还能够通过文本优化进一步提升广告的个性化适配度。

另一方面,在神经质维度上,实验1和实验2均未能发现预期的个性化效果,且在实验2中,所有创作者的广告均表现出稳定的负向匹配效应,即神经质水平较低的个体对针对高神经质设计的广告评价更高。这一现象可能表明,神经质个体在广告偏好上的特征较为特殊,并不容易通过传统的个性化匹配策略进行有效优化。一个可能的解释是,高神经质个体通常具有较高的焦虑和不安全感,因此他们可能更倾向于接受带有安抚性、稳定性描述的广告,而不是直接迎合其负面情绪特征的广告。而低神经质个体可能对“高神经质广告”中涉及的情绪化表达、焦虑驱动的语言特征更敏感,甚至产生更强的情绪共鸣,从而导致负向匹配效应的出现。此外,高神经质个体可能对广告信息的解读方式不同于低神经质个体,他们更容易受到信息的情绪基调和潜在风险暗示的影响这可能进一步导致高神经质个体对高神经质广告的接受度较低,而低神经质个体则因广告的情绪化特征反而产生更高的评价。这一发现提示,在未来的个性化广告优化中,需要更深入地探讨神经质个体对广告内容的具体偏好模式,以避免简单的人格匹配策略在这一特质上的局限性。

尽管研究一和研究二验证了AI在个性化广告创作中的有效性,并探索了其与人类专家在个性化匹配效果上的差异,但这些实验的分析主要基于参与者被试的行为反馈,而未能深入解析广告文本本身的特征。特别是在尽责性和宜人性维度,AI展现出更稳定且更优的个性化匹配能力,而人类专家创作的广告则表现出较大的不确定性,这一结果表明,个性化广告的效果不仅受到受众人格特质的影响,也可能取决于广告文本本身的语言模式。然而,目前的实验结果仍未能清楚地揭示个性化广告创作中所采用的文本特征究竟存在何种系统性的差异,以及这些文本特征如何影响受众对广告的接受度。因此,仅通过参与者评分来判断个性化广告的有效性仍然是不够的,进一步的文本分析有助于揭示广告文本本身的结构、内容以及语言特征如何与不同人格特质的受众需求相匹配。在个性化广告研究中,文本特征的作用往往难以被系统地研究,因为传统的广告文本数量较少,难以进行规模化的文本分析。然而,AI的快速生成能力使得本研究能够积累大量的个性化广告文本及其参与者评分数据,从而为文本特征的深入分析提供了可能性。研究三将在研究一和研究二的基础上,结合大规模文本数据,系统地分析个性化广告文本在不同人格特质维度上的语言模式,并构建预测模型,以识别最能影响个性化广告说服效果的关键特征。这不仅能够帮助我们更好地理解AI在个性化广告创作中的优势与局限,也能为未来的人机协作广告创作提供数据驱动的优化策略,从而推动个性化广告生成技术的进一步发展。