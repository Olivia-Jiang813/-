\section{讨论} 

本研究通过文本分析和预测建模,深入探讨了个性化广告文本的语言特征及其对广告说服效果的影响。研究三的结果表明,不同人格特质的个性化广告在语言风格上存在系统性差异 \citep{koutsoumpis2022kernel},且这些差异能够有效预测广告的个性化说服效果。此外,本研究进一步比较了 AI 生成的个性化广告文本特征 与 不同人格特质个体在实际偏好中表现出的语言特征,揭示了 AI 在个性化广告生成中的局限性。尽管 AI 在广告设计时基于各特质的语言风格进行针对性优化,但由于语言使用在不同情境下可能呈现不同特征,即使是同一人格特质,针对不同广告类型,其语言风格也可能存在显著差异 \citep{flores2014effect},AI 在个性特质与广告场景的结合方面仍然存在短板。此外,以往研究主要关注高水平特质的语言偏好,而对低水平特质的个性化偏好探讨较少,本研究的结果表明,低水平特质个体的广告偏好并不完全符合传统人格特质的解读,而在广告情境下表现出新的特征倾向。例如,某些低水平特质个体偏好的特征与其在一般语言研究中的表现不完全一致,这表明个性化广告的说服机制可能受到额外的语境因素影响。这些发现不仅反映了 AI 生成广告在低水平特质个性化匹配上的挑战,也为未来个性化广告优化提供了新的方向。因此,未来的个性化广告生成需要更加关注语言风格在不同语境中的适应性,并结合高低水平特质的实际偏好,以提高个性化匹配的精准度和广告的说服力。接下来将从各人格特质展开,进一步讨论 AI 生成文本与实际个性化偏好的具体差异,并探讨其可能的成因及优化方向。

在\textbf{外倾性}个性化广告的设计中(如表 \ref{tab:study3_extraversion_comparison}),AI 主要关注的是社交互动、情感表达和积极体验,因此生成的广告文本往往包含大量社交相关(social)和正面情感(posemo)词汇,例如“与世界分享你的热情,记录美好时刻,激发他人的快乐与欢笑”。此外,外倾性广告通常也会使用感知类(percept)词汇,以增强体验感。然而,从外倾性参与者的实际反馈来看,尽管这些元素符合外倾性个体的语言风格,但更能提升个性化广告效果的特征还包括激励性表达(reward)和互动性用语(you),如“你的朋友已为此狂热,这不仅是手机,更是社交的新纪元”。相比之下,AI 在生成广告时虽然强调了社交主题,但可能低估了“互动”在广告场景中的重要性。造成这一偏差的一个可能原因是,大多数关于人格与语言的研究侧重于个体的自然语言表达习惯,而不是他们在广告情境下的偏好。外倾性个体的日常语言确实包含更多社交词汇,但这些表达通常是针对社交情境,而非广告语境。在广告场景中,外倾性个体可能更倾向于被直接卷入对话,需要更多“你”(you)的参与感,而不仅仅是对“社交”概念的描述。因此,AI 在广告生成时,应考虑如何从社交语言风格转向“交互式”语言设计,例如采用问题导向、邀请式的表达(如“你将如何使用这款手机?”),以进一步增强广告的吸引力。此外,低外倾性个体的广告偏好与传统外倾性/内倾性语言风格的理解并不完全一致。通常,低外倾性个体在自然语言表达中更偏向于内省、个体化、秩序感,但在广告场景下,他们更倾向于强调权力和掌控感的广告,如“精准管理你的时间,高效完成每一个任务”。这可能表明他们对社交互动的关注度较低,而更希望广告提供增强个人自主性的承诺。这与研究发现的低外倾性个体更喜欢反思、更注重个体控制,并倾向于结构化和秩序化的生活一致 \citep{beukeboom2013language}。这一发现说明,在个性化广告的设计上,AI 不应仅停留在基于人格特质的表层语言特征,而应深入挖掘不同人格特质的更深层次需求,特别是在特定广告情境下如何有效激发不同群体的兴趣和认同感。


\begin{table}[H]
    \centering
    \caption{\label{tab:study3_extraversion_comparison} 高外倾与低外倾的个性化广告语言特征偏好}
    {\tablesongti % 整个表格环境应用宋体六号字体
    \renewcommand{\arraystretch}{1.5} % 调整行距
    \begin{tabularx}{\linewidth}{l X X} % 设定每列宽度
        \toprule
        \textbf{外倾性} & \textbf{AI 生成} & \textbf{实际偏好} \\
        \midrule
        \multirow{5}{*}{\textbf{高外倾}} 
        & 社交相关(social),如“与世界分享你的激情” & 社交相关(social),如“你的朋友已为此狂热” \\
        & 正面情感(posemo),如“让你的社交世界明亮而充满希望” & 正面情感(posemo),如“激发他人的快乐与欢笑” \\
        & 感知体验(percept),如“感受科技带来的精彩” & 感知体验(percept),如“探索全新体验” \\
        &  & \textcolor{red}{互动性(you),如“你将如何使用?”} \\
        &  & \textcolor{red}{激励性表达(reward),如“开启你的无限可能”} \\
        \midrule
        \multirow{5}{*}{\textbf{低外倾}} 
        & \multirow{5}{*}{样本不足未分析}  & 强调个人自主(individuality),如“精准管理你的时间” \\
        &  & 掌控感(control),如“助你高效决策” \\
        &  & 秩序、结构化表达(order),如“让你的生活井然有序” \\
        &  & 权力(power),如“主宰你的生活节奏” \\
        \bottomrule
    \end{tabularx}
    }
\end{table}


在\textbf{宜人性}个性化广告的设计上(如表 \ref{tab:study3_agreeableness_comparison}),AI 主要强调温暖、关怀和归属感,因此生成的广告往往包含较多积极情感(posemo)、社交相关(social)和归属动机(affiliation)词汇,如“用这款手机,把温情传递给每一个需要的角落”。然而,实际广告偏好表明,他们还更加看重比较性表达(compare)和未来导向(focusfuture),例如“精准生活,从不间断”。这一点可能源于高宜人性个体在社交互动中不仅关注情感连接,还关注相对优势和长远利益,他们希望看到广告内容如何凸显产品在人际关系或个人体验中的独特性。这一偏好与以往研究对宜人性个体的理解有所不同,因为过往研究多强调宜人性与和谐沟通的关系,而本研究的结果表明,在广告场景中,高宜人性个体更关注产品的竞争力,以及产品如何带来持续性的价值。而对于低宜人性个体,在实际偏好中更倾向于直接、批判性更强的语言风格,如否定性表达(negate)和愤怒(anger)。此外,他们还对生物相关内容(bio) 产生了更高的偏好,这表明低宜人性个体可能更关注产品对个人生理需求的直接影响,而不是对社交关系的促进。这与传统研究强调低宜人性个体更直接较少关注情感交流和他人感受的特征一致 \citep{crowe2018uncovering}。未来的个性化广告设计应考虑,针对低宜人性个体,广告可以减少冗余的情感表达,而更加强调产品的物理性能、个人体验和问题解决能力,例如“强劲性能,不再忍受卡顿”或“守护你的健康,关心你的身体”。

\begin{table}[H]
    \centering
    \caption{\label{tab:study3_agreeableness_comparison} 高宜人性与低宜人性的个性化广告语言特征偏好}
    {\tablesongti % 整个表格环境应用宋体六号字体
    \renewcommand{\arraystretch}{1.5} % 调整行距
    \begin{tabularx}{\linewidth}{l X X} % 设定每列宽度
        \toprule
        \textbf{宜人性} & \textbf{AI 生成} & \textbf{实际偏好} \\
        \midrule
        \multirow{4}{*}{\textbf{高宜人性}} 
        & 积极情感(posemo),如“用这款手机,把温情传递给每一个需要的角落” & 积极情感(posemo),如“用这款手机,把温情传递给每一个需要的角落” \\
        & 社交相关(social),如“记录美好瞬间,让善意的温暖传递到每个角落” & 社交相关(social),如“记录美好瞬间,让善意的温暖传递到每个角落” \\
        & 归属动机(affiliation),如“与世界分享你的温暖,让每一次沟通都更有意义” &  \textcolor{red}{比较(compare),如“精准生活,从不间断”} \\
        & &  \textcolor{red}{未来导向(focusfuture),如“开启你的无限可能”}  \\
        \midrule
        \multirow{4}{*}{\textbf{低宜人性}} 
        & \multirow{4}{*}{样本不足未分析} \\
        & & 直接性表达(directness),如“告别卡顿死机” \\
        & & 否定(negate),如“再也不用担心电池续航” \\
        & & 愤怒(anger),如“强劲性能,不再忍受卡顿” \\
        & & 生物相关(bio),如“守护你的健康,关心你的身体” \\
        \bottomrule
    \end{tabularx}
    }
\end{table}


在\textbf{尽责性}的个性化广告设计中 (\ref{study3_conscientiousness_comparison}),AI 主要强调目标导向、条理性和成就感,因此生成的广告文本通常包含较多成就动机(achieve)、数量表达(quant) 以及长词(sixltr),如“精准管理你的日常,高效提升你的生产力。” 这些特征确实符合高尽责性个体的语言风格,但从实际偏好来看,他们更倾向于逻辑清晰的功能描述(function) 和 未来导向表达(focusfuture),如“掌控你的时间,高效完成每一个任务。” 这一发现表明,高尽责性个体不仅关注广告是否强调成就和效率,还更希望广告能提供清晰的目标实现路径。然而,AI 在广告生成时通常更偏向于静态描述,强调“提升效率”或“提高生产力”,而忽略了如何实现这一目标的具体行动步骤。造成这一偏差的一个可能原因是,AI 主要基于语言风格匹配,而非深度建模尽责性个体的思维模式。尽责性个体通常偏好结构化的因果逻辑,即如何从当前状态迈向未来目标,但 AI 生成的广告语料主要受商业营销策略影响,过度依赖数据支持(quant)和成就动机(achieve),而缺乏因果推理(cause)。例如:AI 当前生成:“每天提升 20\% 的工作效率。” 高尽责性个体可能更偏好:“通过智能计划和任务提醒,帮助你每天提升 20\% 的工作效率。” 此外,分析得到低尽责性个体偏好的广告特征是带有紧迫感或挑战性的广告,如“如果你不立即采取行动,你可能会错失机会”。这与低尽责性个体的心理特质相一致,他们往往缺乏长期规划能力,更容易受到即时刺激和情境驱动的影响 \citep{deyoung2010impulsivity}。

\begin{table}[H]
    \centering
    \caption{\label{tab:study3_conscientiousness_comparison} 高尽责与低尽责的个性化广告语言特征偏好}
    {\tablesongti % 整个表格环境应用宋体六号字体
    \renewcommand{\arraystretch}{1.5} % 调整行距
    \begin{tabularx}{\linewidth}{l X X} % 设定每列宽度
        \toprule
        \textbf{尽责性} & \textbf{AI 生成} & \textbf{实际偏好} \\
        \midrule
        \multirow{5}{*}{\textbf{高尽责性}} 
        & 目标导向(achieve),如“精确管理你的日常,高效提升你的生产力” & \textcolor{red}{逻辑清晰的功能描述(function),如“通过智能计划和任务提效,帮助你每天提高 20\% 的工作效率”} \\
        & 数量表达(quant),如“提升 20\%的效率” & \textcolor{red}{未来导向表达(focusfuture),如“掌控你的时间,高效完成每一个任务”} \\
        & 长词使用(sixltr),语言更正式、结构化 & \textcolor{red}{因果推理(cause),如“高效计划每一天,从目标设定到成果实现”}\\
        \midrule
        \multirow{5}{*}{\textbf{低尽责性}} 
        & \multirow{5}{*}{样本不足未分析} \\
        &  & 紧迫感(urgency),如“如果你不立即采取行动,你可能会错失机会”\\
        &  & 挑战性(challenge),如“现在就抓住住机会!”\\
        &  & 非正式表达(informal),如“快来试试,让生活更轻松!”\\
        \bottomrule
    \end{tabularx}
    }
\end{table}


在\textbf{开放性}广告的设计中(表\ref{tab:study3_openness_comparison}),AI 主要关注探索性、创造性和认知加工,因此生成的广告往往包含较多认知加工(cogproc)、洞察(insight)和空间感知(space)词汇,如“探索这款手机,感受尖端科技与前卫设计的融合”。这些特征符合开放性个体在语言使用上更倾向于抽象概念和认知探索的特点 \citep{deyoung2014openness},然而,实际个性化广告的测试结果显示,因果逻辑(cause)和非正式表达(informal)可能在广告说服过程中更具影响力,例如“你的世界,由你定义”这样的表达比单纯的探索性描述更有效。值得注意的是,因果逻辑的偏好与前人文献的结论相反\citep{pennebaker1999linguistic}。以往研究认为,高开放性个体的语言风格更倾向于发散性思维,具有更高的抽象性和创造性,因而可能更偏好非线性、开放式的表达。然而,本研究的发现表明,在广告语境中,当广告内容涉及个性化说服时,高开放性个体更容易关注信息的逻辑性,而非单纯的抽象表达。这可能是因为广告本质上是一种说服性沟通,个性化广告的匹配过程可能会促使受众进行更深层次的认知加工(deliberate processing),而在这一过程中,因果关系明确的句子能够提供清晰的推理路径,使高开放性个体更容易理解广告的核心价值。例如,相较于“解锁你的创造力,探索无限可能”这种较为模糊的表达,高开放性个体在广告评价中更青睐“通过 AI 影像增强技术,让你的创意变得更清晰”,因为后者提供了明确的因果关系,增强了信息的可理解性和可信度。此外,本研究还发现,高开放性个体在广告中更偏好非正式表达(informal),这一点与高开放性个体更倾向于接受个性化、随性的表达方式的特点一致。研究表明,在创意广告中,带有幽默或非正式元素的广告更能够激发灵活思维和创造力 \citep{kover1995creativity},更符合高开放性个体对新颖性和独特性的偏好。例如,“嘿,你准备好迎接一款真正与众不同的手机了吗?” 这样的非正式语句可能比“本产品将带来创新性的使用体验”更能引起高开放性个体的兴趣。相比之下,AI 生成的广告文本更多采用正式、标准化的表达,而在未来的优化方向中,可以进一步融入更加个性化、随性且富有创造力的非正式表达,以提升对高开放性个体的吸引力。在低开放性个体的个性化偏好中,本研究发现他们更倾向于数据驱动和事实支撑的信息,例如“99\% 的用户推荐”或“实验数据显示,这款产品性能提升 20\%”;以及低对确定性(certainty)和稳固性(stability)的需求,例如“经过验证,这款手机更加稳定可靠”,这与低开放性更倾向于可预测、可靠产品的特点相符。

\begin{table}[H]
\centering
\caption{\label{tab:study3_openness_comparison} 高低开放性个体的AI 生成特征与实际偏好特征对比}
{\tablesongti
\renewcommand{\arraystretch}{1.5} % 调整行距
\begin{tabularx}{\linewidth}{l X X} % 设定每列宽度
    \toprule
    \textbf{开放性} & \textbf{AI 生成} & \textbf{实际偏好} \\
    \midrule
    \multirow{4}{*}{\textbf{高开放性}} 
    & 认知加工(cogproc),如“探索这款手机,感受尖端科技与前卫设计的融合” & 认知加工(cogproc),如“你的世界,由你定义” \\
    & 洞察(insight),如“打开新的可能,创造属于你的未来” & \textcolor{red}{因果逻辑(cause),如“先进技术让使用体验更加流畅”} \\
    & 空间感知(space),如“广阔视野,让探索无界” & \textcolor{red}{非正式表达(informal),如“随时随地,想拍就拍”} \\
    &  & \textcolor{red}{集体归属(we),如“共同追求进步,创造美好的明天”} \\
    \midrule
    \multirow{4}{*}{\textbf{低开放性}} 
    & \multirow{4}{*}{样本不足未分析}\\
    &  & 数据驱动(number),如“99\% 的用户推荐” \\
    &  & 确定性(certainty),如“经过验证,这款手机更加稳定可靠” \\
    &  & 稳固性(stability),如“专业测试认证,确保长期使用稳定” \\
    \bottomrule
\end{tabularx}
}
\end{table}


在\textbf{神经质}个性化广告的设计中(如表\ref{tab:study3-neuroticism_comparison}),AI 主要关注情绪波动和安全感,因此生成的广告往往包含较多负面情感(negemo)、焦虑(anx)、愤怒(anger)和悲伤(sad)词汇,如“在泪水中,她是你疗愈伤痕的港湾”。这些表达与高神经质个体的典型语言风格一致,符合他们对情绪相关信息更敏感的特点。然而,实际偏好分析显示,高神经质个体更倾向于非正式表达(netspeak)和即时满足(reward),如“在快节奏的生活中,找到一个真正与你同步的伴侣”。相比之下,AI 低估了这些因素在个性化广告中的作用。这可能源于神经质个体在自然语言表达和广告接受偏好上的差异——虽然他们的语言风格通常表现出高情绪性,但他们在广告中更偏好具有亲和力、轻松随性的表达方式,而不是强化负面情绪。例如,非正式表达可以减少广告对焦虑个体的威胁性,使广告更具可接受性,而即时满足的承诺则能够降低不确定性,为他们提供即时的情绪安抚 \citep{miller2006neuroticism}。因此,AI 在优化神经质广告时,不应仅关注情绪化表达,而应结合更轻松、互动式的语言,提供更直接的情绪补偿和安全感承诺。在低神经质个体的个性化偏好中,本研究发现他们更倾向于长期规划(focusfuture)和稳定性(stability)的表达,例如“开启智慧生活,助你成为更好的自己”或“经过时间验证,依然强大稳定”。这与他们更低的情绪不稳定性和更高的心理韧性一致。相比高神经质个体的即时满足倾向,低神经质个体更能接受延迟满足,因此强调未来利益的广告对他们更具吸引力。此外,他们也较少受到情绪化信息的影响,更倾向于理性、务实的语言风格。因此,在个性化广告设计中,针对低神经质个体的广告应避免情绪性夸张,而是强调稳定、可靠的产品优势,并通过明确的未来规划信息增强广告的说服力。高低神经质的个性化偏好在过往文献中常无法得到有效的结果,个体的实际偏好差异,表明在广告个性化策略上,不能仅依据神经质个体在自然语言中的情绪表达风格,而应结合广告的实际作用机制进行调整。高神经质个体的广告需要减少焦虑诱发因素,并提供即时情绪安抚,而低神经质个体则更偏好稳定、长期规划的信息结构。未来的个性化广告优化可以进一步结合情绪调节策略,使广告内容不仅与个体语言风格匹配,也能够针对个体心理需求进行精准调整。以往文献发现,高低神经质个体的广告接受偏好并不像其他人格特质那样容易区分 \citep{matz2024potential},部分原因在于高神经质个体不仅关注情绪共鸣,也渴望获得缓解焦虑的方式,而低神经质个体对情绪内容的接受度较高,但更偏好清晰的事实信息。因此,未来的个性化广告优化应进一步结合个体的情绪调节需求,针对高神经质个体减少焦虑诱发因素,提供即时满足的信息,而针对低神经质个体则强调长期规划和稳定性,使广告内容能够更精准地适应不同个体的心理需求。

\begin{table}[H]
\centering
\caption{\label{tab:study3-neuroticism_comparison} 高低神经质个体的AI 生成特征与实际偏好特征对比}
{\tablesongti
\renewcommand{\arraystretch}{1.5} % 调整行距
\begin{tabularx}{\linewidth}{l X X} % 设定每列宽度
    \toprule
    \textbf{神经质} & \textbf{AI 生成} & \textbf{实际偏好} \\
    \midrule
    \multirow{2}{*}{\textbf{高神经质}} 
    & 负面情感(negemo),如“在泪水中,她是你疗愈伤痕的港湾” & \textcolor{red}{非正式表达(netspeak),如“在快节奏的生活中,找到一个真正与你同步的伴侣”} \\
    & 焦虑(anx)、愤怒(anger)、悲伤(sad)词汇,突出情绪波动 & \textcolor{red}{即时满足(reward),如“开启无限可能,即刻体验”} \\
    \midrule
    \multirow{2}{*}{\textbf{低神经质}} 
    &  \multirow{2}{*}{样本不足未分析} \\
    &  & 长期规划(focusfuture),如“开启智慧生活,助你成为更好的自己” \\
    &  & 稳定性(stability),如“经过时间验证,依然强大稳定” \\
    \bottomrule
\end{tabularx}
}
\end{table}


本研究通过对 AI 生成的个性化广告文本与不同人格特质个体的实际偏好进行比较,揭示了 AI 在广告生成中的局限性:语言使用的场景适应性不足、个性化策略的刻板化。尽管 AI 生成的广告在一定程度上能够匹配目标特质的语言风格,但语言风格的匹配并不必然等同于广告的高效说服力。广告作为一种具有特定传播目的的信息载体,其个性化策略不仅需要精准表达目标特质,还需兼顾广告语境的适应性,以及受众在信息加工过程中的认知与情感需求。针对这一问题,未来 AI 生成个性化广告的优化方向需要从单纯的语言风格匹配拓展至更深层次的语境适配性。具体而言,AI 在个性化广告生成时,不应仅关注受众在自然语言中的表达习惯,还需识别特定广告场景下语言使用的实际功能。例如,即便某些语言特征在日常交流中较为常见,它们在广告语境下的有效性仍可能受到受众信息处理模式的影响。不同广告目标(如品牌塑造、产品推广、情感共鸣等)对语言风格的要求不同,AI 需要在个性化表达与广告实际传播效果之间找到平衡,使个性化不仅体现在语言特征的匹配上,更体现在广告说服力的提升上。这不仅为个性化广告优化提供了新的实证支持,也为 AI 在广告创作中的应用提出了新的优化方向。