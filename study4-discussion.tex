\section{讨论}

本研究探讨了AI作为个性化广告信息源时对个性化广告效果的影响,并揭示了信息来源在个性化广告的说服过程中所起的关键作用。研究结果表明,个性化广告的有效性受到感知相似性的中介作用,即当受众认为广告内容与自身特质相匹配时,其说服效果更强 \citep{li2016does, teeny2021review}。然而,当 AI 作为广告创作者的身份被明确标注 时,即使广告文本本身符合目标受众的人格特质,其说服力仍然受到削弱。这一发现表明,个性化广告的有效性不仅取决于内容本身,还受到受众对信息来源的认知影响。

在 信息来源认知 方面,研究结果显示,在信息来源未知的情况下,受众能够基于文本特征推测广告创作者,并且 AI 生成的广告在语言风格上与人类专家撰写的广告表现出系统性差异。然而,当信息来源被明确标注为 AI 时,个性化广告的说服力下降,受众对广告的信任度也随之降低。这一现象可以通过 感知相似性在个性化广告中的中介作用 来解释。AI 作为信息源会增加受众与广告之间的心理距离 \citep{kim2020artificial, ahn2021ai},导致受众在认知上将 AI 视为一个较远的主体,从而降低他们与广告的心理契合度,并最终削弱个性化广告的整体效果。

这一发现对 AI 生成个性化广告的优化策略 具有重要启示。首先,AI 生成广告的优化不应仅停留在匹配目标人格特质的语言风格,而应进一步增强广告的“人性化”表达,以提升其交互性和情感共鸣。例如,研究表明,使用第一人称(I/we)和直接面向受众的语言(you)能够增强个性化体验,减少 AI 生成广告与受众之间的心理距离 \citep{markowitz2020communicating}。这种语言策略能够让广告显得更具互动性,使受众更容易产生自我联结,进而提升个性化广告的说服力。

其次,在 AI 生成广告的披露策略 上,应探索 降低 AI 作为信息来源可能带来的负面影响。研究发现,当 AI 生成的内容被描述为“结合专家意见”或“基于大数据精准分析”时,其可信度和接受度会有所提升 \citep{puerta2022human}。因此,在政策可能要求 AI 生成广告必须披露信息来源 的情况下,企业可以调整披露方式,例如 强调 AI 在广告创作中的辅助角色,而非主要创作者,或通过 人机共创(human-in-the-loop)模式 让人类专家优化 AI 生成的广告内容。这一策略不仅可以结合 AI 的 高效性和大规模数据处理能力,还能够 保留人类专家的创造力和情感共鸣能力,使 AI 生成的个性化广告在 提高传播效率的同时,仍然具备较强的个性化吸引力。
